\chapter{Stein's method}\label{C:steins-method}

In this chapter we present an introduction to Stein's method of
exchangeable pairs which we use to prove the core theoretical result
of this thesis: a rate of convergence bound for the randomization
distribution.

\section{Approximate computation of
  expectations}\label{S:approx-computation}



% \chapter{Multivariate statistics background}\label{C:multivariate-background}

% Multivariate statistics will prove to be a central tool for this
% thesis.  We use this chapter to gather the relevant definitions and
% results, concentrating mainly on the eigenvalues and eigenvectors from
% sample covariance matrices.  The literature in this area spans over
% fifty years.  We give the basic definitions and properties of the
% multivariate normal and Wishart distributions in
% Section~\ref{S:mulitivariate-definitions}.  Then, in
% Section~\ref{S:multivariate-classical}, we survey classical results
% about sample covariance matrices when the number of dimensions, $p$,
% is fixed and the sample size, $n$, grows to infinity.  This material
% is by now standard and can be found in any good multivariate
% statistics book (e.g. Muirhead~\cite{muirhead1982ams}).  Lastly, in
% Section~\ref{S:multivariate-modern}, we survey modern asymptotics,
% where $n \to \infty$ and $p$ grows with $n$.  Modern multivariate
% asymptotics is still an active research topic, but today it is
% possible to give a reasonably-complete description of the objects of
% interest.

% \section{The multivariate normal and Wishart distributions}\label{S:mulitivariate-definitions}

% We start with the definition of the multivariate normal distribution and some basic properties, which can be found, for example, in Muirhead~\cite{muirhead1982ams}[Chapters 1--3].

% \begin{definition}[Multivariate Normal Distribution]
%     \label{D:multivariate-normal}
% For mean vector 
% \(
%     \vmu \in \reals^p
% \)
% and positive-semidefinite covariance matrix
% \(
%     \mSigma \in \reals^{p\times p},
% \)
% a random vector 
% \(
%     \vX \in \reals^p
% \)
% is said to be distributed from the multivariate normal distribution, denoted 
% \(
%     \vX 
%     \sim 
%     \Normal \left(
%         \vmu, \,
%         \mSigma
%     \right)
% \)
% if for every fixed vector $\va \in \reals^p$, the vector
% $\va^\trans \vX$ has a univariate normal distribution with mean
% \(
%     \va^\trans \vmu
% \)
% and variance
% \(
%     \va^\trans \mSigma \va
% \).
% \end{definition}

% \noindent
% The multivariate normal distribution is defined for any positive-semidefinite covariance matrix $\mSigma$, but it only has a density when $\mSigma$ is strictly positive-definite.

% \begin{proposition}
% If $\vX \in \reals^p$ follows a multivariate normal distribution with mean $\vmu$ and positive-definite covariance matrix $\mSigma$, then its components have density
% \begin{equation}\label{E:normal-density}
%     f( \vx )
%     =
%     (2 \pi )^{-p/2}
%     |\mSigma|^{-1/2}
%     \exp\left(
%         -
%         \frac{1}{2}
%         (\vx - \vmu)^\trans
%         \mSigma^{-1}
%         (\vx - \mu)
%     \right).
% \end{equation}
% \end{proposition}

% A basic fact about the multivariate normal is the following:

% \begin{proposition}\label{P:scale-shift-normal}
% Let 
% \(
%     \va \in \reals^p,
% \)
% \(
%     \mC \in \reals^{q \times p},
% \)
% and 
% \(
%     \vX 
%     \sim
%     \Normal \left( 
%         \vmu, \,
%         \mSigma
%     \right).
% \)
% Define $\vY = \mC \vX + \va$.  Then 
% \(
%     \vY
%     \sim
%     \Normal \left( 
%         \mC \vmu + \va, \,
%         \mC \mSigma \mC^\trans
%     \right).
% \)
% \end{proposition}

% \noindent
% Two immediate corollaries are:

% \begin{corollary}\label{C:normal-orthog-invariant}
% Suppose that
% \(
%     \vX 
%     \sim 
%     \Normal \left( 
%         \vzero, \,
%         \sigma^2 \mI_p
%     \right)
% \)
% and that
% \(
%     \mO \in \reals^{p \times p}
% \)
% is an orthogonal matrix.  Then
% \(
%     \mO \vX \eqd \vX.
% \)
% \end{corollary}

% \begin{corollary}
% If
% \(
%     \vX 
%     \sim 
%     \Normal \left( 
%         \vzero, \,
%         \mSigma
%     \right)
% \)
% and $\mSigma = \mC \mC^\trans$ for a matrix $\mC \in \reals^{p \times p}$,
% and if
% \(
%     \vZ
%     \sim
%     \Normal \left(
%         \vzero, \,
%         \mI_p
%     \right),
% \)
% then
% \(
%     \mC \vZ
%     \eqd
%     \vX
% \).
% \end{corollary}    

% We are often interested in estimating the underlying parameters from
% multivariate normal data.  The sufficient statistics are the standard
% estimates.

% \begin{proposition}\label{P:normal-sufficient-stats}
% Say that
% \(
%     \vX_1, \vX_2, \ldots, \vX_n
% \) 
% are independent draws from a
% \(
%     \Normal \left(
%         \vmu, \,
%         \mSigma
%     \right)
% \)
% distribution.  Then the sample mean
% \begin{equation}\label{E:sample-mean}
%     \vbX_n
%     \define
%     \frac{1}{n}
%     \sum_{i=1}^n
%         \vX_i
% \end{equation}
% and the sample covariance
% \begin{equation}\label{E:sample-covariance}
%     \mS_n
%     \define
%     \frac{1}{n-1}
%     \sum_{i=1}^n
%         \left(
%             \vX_i - \vbX_n
%         \right)
%         \left(
%             \vX_i - \vbX_n
%         \right)^\trans
% \end{equation}
% are sufficient statistics for $\vmu$ and $\mSigma$.
% \end{proposition}

% To describe the distribution of $\mS_n$, we need to introduce the 
% Wishart distribution.

% \begin{definition}[Wishart Distribution]\label{D:wishart}
% Let $\vX_1, \vX_2, \ldots, \vX_n \in \reals^p$ be an iid sequence of
% random vectors, each distributed as
% \(
%     \Normal \left(
%         \vzero, \,
%         \mSigma
%     \right).
% \)
% Then the matrix
% \[
%     \mA = \sum_{i=1}^n \vX_i \vX_i^\trans
% \]
% is said to have the Wishart distribution with $n$ degrees of freedom and scale parameter $\mSigma$.  We denote this by
% \(
%     \mA
%     \sim
%     \Wishart_p \left(
%         n, \,
%         \mSigma
%     \right).
% \)
% \end{definition}

% \noindent
% When $n \geq p$ and $\mSigma$ is positive-definite, the elements of a Wishart matrix have a density.
% \begin{proposition}
% Suppose that
% \(
%     \mA
%     \sim
%     \Wishart_p \left(
%         n, \,
%         \mSigma
%     \right)
% \).
% If $n \geq p$ and $\mSigma$ is positive-definite, then the elements of $\mA$ have a density over the space of positive-definite matrices, given by
% \begin{equation}\label{E:wishart-density}
%     f( \mA )
%     =
%     \frac{ |\mA|^\frac{n-p-1}{2} }
%          { 2^\frac{np}{2} 
%            |\mSigma|^\frac{n}{2} 
%            \Gamma_p \left( \frac{n}{2} \right) }
%     \exp \left(
%         -
%         \frac{1}{2}
%         \tr \left(
%             \mSigma^{-1} \mA
%         \right)
%     \right),
% \end{equation}
% where
% \(
%     \Gamma_p \left( \cdot \right)
% \)
% is the multivariate gamma function, computed as
% \begin{equation}
%     \Gamma_p \left( \frac{n}{2} \right)
%     =
%     \pi^{p(p-1)/4}
%     \prod_{i=1}^p
%         \Gamma \left(
%             \frac{n + 1 - i}{2}
%         \right).
% \end{equation}
% \end{proposition}

% We can now characterize the distributions of the sufficient statistics of a sequence of iid multivariate normal random vectors.

% \begin{proposition}
% Let 
% \(
%     \vbX_n
% \) 
% and 
% \(
%     \mS_n
% \)
% be defined as in Proposition~\ref{P:normal-sufficient-stats}.  Then
% \(
%     \vbX_n
% \)
% and
% \(
%     \mS_n
% \)
% are independent with
% \(
%     \vbX_n
%     \sim
%     \Normal \left(
%         \mu, \,
%         \frac{1}{n}
%         \mSigma
%     \right)
% \)
% and
% \(
%     (n-1)
%     \mS_n
%     \sim
%     \Wishart_p \left(
%         n-1, \,
%         \mSigma
%     \right)
% \).
% \end{proposition}

% White Wishart matrices---those with scale parameter $\mSigma = \sigma^2 \mI_p$
% are of particular interest.  We can characterize their distribution in
% terms of eigenvalues and eigenvectors.

% \begin{proposition}
% Suppose that
% \(
%     \mA
%     \sim
%     \Wishart_p \left(
%         n, \,
%         \sigma^2 \mI_p
%     \right)
% \)
% with
% \(
%     n \geq p
% \)
% and let
% \(
%     \mA = n \mO \mL \mO^\trans
% \)
% be the spectral decomposition of
% \(
%     \mA,
% \)
% with
% \(
%     \mL
%     =
%     \diag \left(
%         l_1,
%         l_2,
%         \ldots,
%         l_p
%     \right)
% \)
% and
% \(
%     l_1
%     >
%     l_2
%     >
%     \cdots
%     >
%     l_p
%     >
%     0.
% \)
% Then $\mO$ and $\mL$ are independent, with $\mO$ Haar-distributed over
% the group of $p \times p$ orthogonal matrices and the elements of $\mL$
% having density
% \begin{equation}\label{E:wishart-eig-density}
%     \left(
%         \frac{1}{2 \sigma^2 }
%     \right)^{np/2}
%     \frac{ \pi^{p^2/2} }
%          { \Gamma_p \left( \frac{n}{2} \right) 
%            \Gamma_p \left( \frac{p}{2} \right) }
%     \prod_{i < j}^p
%         \left| l_i - l_j \right|
%     \prod_{i=1}^p
%         l_i^{(n-p-1)/2}
%         e^{-\tfrac{l_i}{2 \sigma^2}}
%         .
% \end{equation}    
% \end{proposition}
% \noindent
% In the random matrix theory literature, with $\sigma^2 = 1$ the eigenvalue density above is sometimes referred to as the Laguerre Orthogonal Ensemble (LOE).

% \section{Classical asymptotics}\label{S:multivariate-classical}

% In this section we present results about sample covariance matrices when the sample size, $n$, tends to infinity, with the number of dimensions, $p$ a fixed constant.  A straightforward application of the strong law of large numbers gives us the limits of the sample mean and covariance.

% \begin{proposition}\label{P:suff-stat-limits}
% Let $\vX_1, \vX_2, \ldots, \vX_n$ be a sequence of iid random vectors in $\reals^p$ with
% \(
%     \E \left[
%         \vX_1
%     \right]
%     =
%     \vmu
% \)
% and
% \(
%     \E \left[
%         \left( \vX_1 - \vmu \right)
%         \left( \vX_1 - \vmu \right)^\trans
%     \right]
%     =
%     \mSigma.
% \)
% Then, as $n \to \infty$,
% \[
%     \vbX_n
%     \define
%     \frac{1}{n}
%     \sum_{i=1}^n
%         \vX_i
%     \toas
%     \vmu
% \]
% and
% \[
%     \mS_n
%     \equiv
%     \frac{1}{n-1}
%     \sum_{i=1}^{n}
%         \left( \vX_i - \vbX_n \right)
%         \left( \vX_i - \vbX_n \right)^\trans
%     \toas
%     \mSigma.
% \]
% \end{proposition}

% To simplify matters, for the rest of the section we will mostly work in a setting when the variables have been centered.  In this case, the sample covariance matrix takes the form
% \(
%     \mS_n = \frac{1}{n} \sum_{i=1}^n \vX_i \vX_i^\trans.
% \)
% To see that centering the variables does not change the theory in any essential way, we provide the following proposition.

% \begin{proposition}
% Let $\vX_1, \vX_2, \ldots, \vX_n$ be a sequence of random observations in $\reals^p$ with mean vector $\vmu$ and covariance matrix $\mSigma$.  Let
% \(
%     \vbX_n = \frac{1}{n} \sum_{i=1}^n \vX_i
% \)
% and
% \(
%     \mS_n
%     =
%     \frac{1}{n-1}
%     \sum_{i=1}^n
%         \left( \vX_i - \vbX_n \right) \!
%         \left( \vX_i - \vbX_n \right)^\trans
% \)
% be the sample mean and covariance, respectively.  Define the centered
% variables
% \(
%     \vtX_i = \vX_i - \vmu.
% \)
% Then
% \[
%     \mS_n
%     =
%     \frac{1}{n}
%     \sum_{i=1}^n
%         \vtX_i \vtX_i^\trans
%     +
%     \OhP\left( \frac{1}{n} \right).
% \]
% In particular, this implies that
% \[
%     \sqrt{n}
%     \left(
%         \mS_n - \mSigma
%     \right)
%     =
%     \sqrt{n}
%     \left(
%         \frac{1}{n}
%         \sum_{i=1}^n
%             \vtX_i \vtX_i^\trans
%         -
%         \mSigma
%     \right)
%     + 
%     \OhP \left( \frac{1}{\sqrt{n}} \right).
% \]
% \end{proposition}
% \begin{proof}
% We can write
% \begin{align*}
%     \mS_n &= \frac{1}{n-1}
%              \sum_{i=1}^n \vtX_i \vtX_i^\trans
%              +
%              \frac{n}{n-1}
%              \left(
%                 \vbX_n - \vmu
%              \right)
%              \left(
%                 \vbX_n - \vmu
%              \right)^\trans \\
%           &= \left(
%                 \frac{1}{n}
%                 +
%                 \frac{1}{n (n-1) }
%              \right)
%              \sum_{i=1}^n \vtX_i \vtX_i^\trans
%              +
%              \frac{n}{n-1}
%              \left(
%                 \vbX_n - \vmu
%              \right)
%              \left(
%                 \vbX_n - \vmu
%              \right)^\trans
% \end{align*}
% The result follows since
% \(
%     \vtX_i \vtX_i^\trans = \OhP \left( 1 \right)
% \)
% and
% \(
%     \vbX_n - \vmu 
%     = 
%     \OhP \left( \frac{1}{ \sqrt{n} } \right).
% \)

% \end{proof}

% The next fact follows directly from the multivariate central limit theorem.

% \begin{proposition}\label{P:sample-cov-limit}
% Suppose that 
% \(
%     \vX_1, \vX_2, \ldots, \vX_n
% \)
% is a sequence of iid random vectors in $\reals^p$ with 
% \[
%     \E \left[
%         \vX_1 \vX_1^\trans
%     \right]
%     =
%     \mSigma,
% \]
% and that for all $1 \leq i,j,i',j' \leq p$ there exists finite
% \(
%     \Gamma_{iji'j'}
% \)
% with
% \[
%     \E \left[
%         \left(
%             X_{1i} X_{1j}
%             -
%             \Sigma_{ij}
%         \right)
%         \left(
%             X_{1i'} X_{1j'}
%             -
%             \Sigma_{i'j'}
%         \right)
%     \right]
%     =
%     \Gamma_{iji'j'}.
% \]
% If
% \(
%     \mS_n
%     =
%     \frac{1}{n}
%     \sum_{i=1}^n
%         \vX_i \vX_i^\trans,
% \)
% then
% \(
%     \sqrt{n}
%     \vecm \left( 
%         \mS_n - \mSigma 
%     \right)
%     \tod
%     \vecm \left( 
%         \mG
%     \right),
% \)
% where $\mG$ is a random $p \times p$ symmetric matrix with 
% \(
%     \vecm \left( \mG \right)
% \)
% a mean-zero multivariate normal having covariance
% \(
%     \cov \left(
%         G_{ij}, \,
%         G_{i'j'}
%     \right)
%     =
%     \Gamma_{iji'j'}.
% \)
% \end{proposition}

% \noindent
% In the previous proposition, $\vecm$ is an operator that stacks the columns of
% an $n\times p$ matrix to create an $n p$-dimensional vector.

% If the elements of $\vX_1$ have vanishing first and third moments (for instance if
% the distribution of $\vX_1$ is symmetric about the origin, i.e.
% \(
%     \vX_1 \eqd - \vX_1
% \)),
% and if
% \(
%     \E \left[
%         \vX_1 \vX_1^\trans
%     \right]
%     =
%     \diag \left(
%         \lambda_1,
%         \lambda_2,
%         \ldots,
%         \lambda_p
%     \right),
% \)
% then $\Gamma_{iji'j'}$ simplifies to
% \begin{subequations}
% \begin{align}
%     \Gamma_{iiii} &= \E \left[ X_{1i}^4 \right] - \lambda_i^2 \quad 
%                          &&\text{for $1 \leq i \leq p$}, \\
%     \Gamma_{ijij} = \Gamma_{ijji} &= \E \left[ X_{1i}^2 X_{1j}^2 \right]
%                          &&\text{for $ 1 \leq i,j \leq p$, $i \neq j$;}
% \end{align}
% \end{subequations}
% all other values of $\Gamma_{iji'j'}$ are $0$.  In particular, this
% implies that the elements of $\vecm \left( \mG \right)$ are independent.
% If we also have that $\vX_1$ is multivariate normal, then 
% \begin{subequations}
% \begin{align}
%     \Gamma_{iiii} &= 2 \lambda_i^2
%                          &&\text{for $1 \leq i \leq p$, and} \\    
%     \Gamma_{ijij} = \Gamma_{ijji} &= \lambda_i \lambda_j
%                          &&\text{for $ 1 \leq i,j \leq p$, $i \neq j$.}
% \end{align}
% \end{subequations}


% It is inconvenient to study the properties of the sample covariance matrix when the population covariance
% \(
%     \mSigma
% \)
% is not diagonal.  By factorizing
% \(
%     \mSigma
%     =
%     \mPhi
%     \mLambda
%     \mPhi^\trans
% \)
% for orthogonal $\mPhi$ and diagonal $\mLambda$, we can introduce
% \(
%     \vtX_i
%     \define
%     \mPhi^\trans \vX_i
% \)
% to get
% \(
%     \E \left[
%         \vtX_i
%         \vtX_i^\trans
%     \right]
%     =
%     \mLambda
% \)
% and
% \(
%     \mS_n
%     =
%     \mPhi
%     \left(
%         \frac{1}{n}
%         \sum_{i=1}^n
%             \vtX_i \vtX_i^\trans
%     \right)
%     \mPhi^\trans.
% \)
% With this transformation, we can characterize the distribution of $\mS_n$ completely in terms of
% \(
%     \frac{1}{n}
%     \sum_{i=1}^n
%         \vtX_i \vtX_i^\trans.
% \)


% The next result we present is about the sample principal components.  It is motivated by Proposition~\ref{P:sample-cov-limit} and is originally due to Anderson~\cite{anderson1963atp}.

% \begin{theorem}\label{T:eigen-random-perturb}
% For $n\to \infty$ and $p$ fixed, let $\mS_1, \mS_2, \ldots, \mS_n$ be a sequence of random symmetric $p \times p$ matrices with 
% \(
%     \sqrt{n} \vecm \left( \mS_n - \mLambda \right)
%     \tod
%     \vecm \left( \mG \right),
% \)
% for a deterministic
% \(
%     \mLambda = \diag\left( \lambda_1, \lambda_2, \ldots, \lambda_p \right)
% \)
% having $\lambda_1 > \lambda_2 > \cdots > \lambda_p$ and a random symmetric matrix $\mG$.
% Let
% \(
%     \mS_n = \mU_n \mL_n \mU_n^\trans
% \)
% be the eigendecomposition of $\mS_n$, with
% \(
%     \mL_n = \diag\left( l_{n,1}, l_{n,2}, \ldots, l_{n,p} \right)
% \)
% and
% \(
%     l_{n,1} \geq l_{n,2} \geq \cdots \geq l_{n,p}.
% \)
% If $\mG = \OhP\left( 1 \right)$, and the signs of $\mU_n$ are chosen so that $U_{n,ii} \geq 0$ for $1 \leq i \leq p$, then the elements of $\mU_n$ and $\mL_n$ converge jointly as
% \begin{subequations}
% \begin{alignat}{2}
%     \sqrt{n}
%     \left( U_{n, ii} - 1 \right)
%         &\toP 0
%                 &&\quad \text{for $1 \leq i \leq p$,} \\
%     \sqrt{n}
%     U_{n, ij}
%         &\tod -
%               \frac{ G_{ij} }{ \lambda_i - \lambda_j }
%                 &&\quad \text{for $1 \leq i,j \leq p$ with $i \neq j$, and} \\
%     \sqrt{n}
%     \left( l_{n,i} - \lambda_i \right)
%         &\tod G_{ii}
%                 &&\quad \text{for $1 \leq i \leq p$.} \qedhere
% \end{alignat}
% \end{subequations}
% \end{theorem}
% \noindent
% More generally, Anderson treats the case when the $\lambda_i$ are not all unique.  The key ingredient to Anderson's proof is a perturbation lemma, which we state and prove below.
% \begin{lemma}\label{L:eigen-perturb}
% For 
% \(
%     n \to \infty
% \)
% and fixed
% \(
%     p
% \)
% let
% \(
%     \mS_1, \mS_2, \ldots, \mS_n \in \reals^{p\times p}
% \) 
% be a sequence of symmetric matrices of the form
% \[
%     \mS_n 
%     = 
%     \mLambda
%     +
%     \frac{1}{\sqrt{n}}
%     \mH_n
%     +
%     \oh\left( \frac{1}{\sqrt{n}} \right),
% \]
% where
% \(
%     \mLambda
%     =
%     \diag \left(
%         \lambda_1,
%         \lambda_2,
%         \ldots,
%         \lambda_p
%     \right)
% \)
% with 
% \(
%     \lambda_1 > \lambda_2 > \cdots > \lambda_p
% \) 
% and
% \(
%     \mH_n = \Oh\left( 1 \right).
% \)
% Let $\mS_n = \mU_n \mL_n \mU_n^\trans$ be the eigendecomposition of $\mS_n$, with
% \(
%     \mL_n
%     =
%     \diag \left(
%         l_{n,1}, l_{n,2}, \ldots, l_{n,p}
%     \right).
% \)
% Further suppose that $U_{n,ii} \geq 0$ for $1 \leq i \leq p$ and
% \(
%     l_{n,1} > l_{n,2} > \cdots > l_{n,p}.
% \)
% Then for all $1 \leq i,j \leq p$ and $i \neq j$ we have
% \begin{subequations}
% \begin{align}
%     U_{n,ii} 
%         &= 1 
%            + 
%            \oh \left( 
%                \frac{ 1 }{ \sqrt{n} }
%            \right), \\
%     U_{n,ij}
%         &= -
%            \frac{ 1 }{ \sqrt{n} }
%            \frac{ H_{n,ij} }{ \lambda_i - \lambda_j }
%            +
%            \oh \left(
%                \frac{ 1 }{ \sqrt{n} }
%            \right), \quad \text{and} \\
%     l_{n,i}
%         &= \lambda_i
%            + 
%            \frac{ H_{n,ii} }{ \sqrt{n} }
%            +
%            \oh \left(
%                \frac{ 1 }{ \sqrt{n} }
%            \right).
% \end{align}
% \end{subequations}
% \end{lemma}
% \begin{proof}
% Define $p\times p$ matrices $\mE_n$, $\mF_n$, and $\mDelta_n$ so that
% \begin{align}
%     \mE_n
%         &=
%         \diag \left(
%             U_{n,11},
%             U_{n,22},
%             \ldots,
%             U_{n,pp}
%         \right), \\
%     \mF_n 
%         &= 
%         \sqrt{n} \left( \mU_n - \mE_n \right), \\
%     \mDelta_{n}
%         &=
%         \sqrt{n} \left( \mL_n - \mLambda \right), \\
% \intertext{giving}
%     \mU_n
%         &= \mE_n + \frac{1}{\sqrt{n}} \mF_n, \quad \text{and} \notag \\
%     \mL_n
%         &= \mLambda + \frac{1}{\sqrt{n}} \mDelta_{n}. \notag
% \end{align}
% We have that 
% \begin{align}
%     \mS_n 
%     &= \mLambda 
%        + \frac{1}{\sqrt{n}} \mH_n 
%        + \oh \left( \frac{1}{\sqrt{n}}\right) \notag \\
%     &= \mU_n \mL_n \mU_n^\trans \notag \\
%     &= \mE_n \mLambda \mE_n^\trans
%        + 
%        \frac{1}{\sqrt{n}} \left(
%            \mE_n \mDelta_n \mE_n^\trans
%            +
%            \mF_n \mLambda \mE_n^\trans
%            +
%            \mE_n \mLambda \mF_n^\trans
%        \right)
%        +
%        \frac{1}{n}
%        \mM_n \label{L:eigen-perturb:E:eigen-perturb}
% \end{align}
% where the elements of $\mM_n$ are sums of $\Oh\left( p \right)$ terms, with each term a product of elements taken from $\mE_n$, $\mF_n$, $\mLambda$, and $\mDelta_n$.  Also,
% \begin{align}
%     \mI_p
%     &= \mU_n \mU_n^\trans \notag \\
%     &= \mE_n \mE_n^\trans
%        + 
%        \frac{1}{\sqrt{n}} \left(
%            \mF_n \mE_n^\trans
%            +
%            \mE_n \mF_n^\trans
%        \right)
%        + 
%        \frac{1}{n}
%        \mW_n, \label{L:eigen-perturb:E:orthog-perturb}
% \end{align}
% where $\mW_n = \mF_n \mF_n^\trans$.
% From \eqref{L:eigen-perturb:E:orthog-perturb} we see that for $1 \leq i,j \leq p$ and $i \neq j$ we must have
% \begin{subequations}
% \begin{align}
%     1 &= E_{n,ii}^2 
%          + 
%          \frac{1}{n} W_{n,ii}, 
%          \quad \text{and}
%          \label{L:eigen-perturb:E:orthog-perturb-1} \\
%     0 &= E_{n,ii} F_{n,ji} 
%          + 
%          F_{n,ij} E_{n,jj} 
%          + 
%          \frac{1}{\sqrt{n}} W_{n,ij}.
%          \label{L:eigen-perturb:E:orthog-perturb-2}
% \end{align}
% \end{subequations}
% Substituting
% \(
%     E_{n,ii}^2 = 1 - \frac{1}{n} W_{n,ii}
% \)
% into equation~\eqref{L:eigen-perturb:E:eigen-perturb}, we get
% \begin{subequations}
% \begin{align}
%     H_{n,ii} 
%         &= E_{n,ii} \Delta_{n,ii} E_{n,ii}
%            + 
%            \frac{1}{\sqrt{n}} \left(
%                 M_{n,ii} - \lambda_i W_{n,ii}
%            \right)
%            +
%            \oh \left( 1 \right), \quad \text{and} 
%            \label{L:eigen-perturb:E:eigen-perturb-1} \\
%     H_{n,ij}
%         &= \lambda_j E_{n,jj} F_{n,ij}
%            +
%            \lambda_i F_{n,ji} E_{n,ii} 
%            +
%            \frac{1}{\sqrt{n}} M_{n,ij}
%            +
%            \oh \left( 1 \right).
%            \label{L:eigen-perturb:E:eigen-perturb-2} 
% \end{align}
% \end{subequations}
% Equations~\eqref{L:eigen-perturb:E:orthog-perturb-1}--\eqref{L:eigen-perturb:E:eigen-perturb-2} admit the solution
% \begin{subequations}
% \begin{align}
%     E_{n,ii} 
%         &= 1 + \oh\left(\frac{1}{\sqrt{n}}\right), \\
%     F_{n,ij}
%         &= -
%            \frac{ H_{n,ij} }
%                 { \lambda_i - \lambda_j }
%            +
%            \oh \left( 1 \right), \quad \text{and} \\
%     \Delta_{n,ii}
%         &= H_{n,ii} + \oh\left( 1 \right). 
% \end{align}
% \end{subequations}
% This completes the proof.
% \end{proof}

% An application of the results of this section is the following theorem, which describes the behavior of principal component analysis for large $n$ and fixed $p$.

% \begin{theorem}
% Let $\vX_1, \vX_2, \ldots, \vX_n$ be a sequence of iid 
% \(
%     \Normal \left(
%         \vmu, \,
%         \mSigma
%     \right)
% \)
% random vectors in $\reals^p$, with sample mean
% \(
%     \vbX_n
% \)
% and sample covariance
% \(
%     \mS_n.
% \)
% Let $\mSigma = \mPhi \mLambda \mPhi^\trans$ be the eigendecomposition
% of $\mSigma$, with
% \(
%     \mLambda = \diag \left(
%         \lambda_1,
%         \lambda_2,
%         \ldots,
%         \lambda_p
%     \right)
% \)
% and
% \(
%     \lambda_1 > \lambda_2 > \cdots > \lambda_p > 0
% \).
% Similarly, let $\mS_n = \mU_n \mL_n \mU_n^\trans$ be the eigendecomposition of $\mS_n$, likewise with
% \(
%     \mL_n = \diag \left(
%         l_{n,1},
%         l_{n,2},
%         \ldots,
%         l_{n,p}
%     \right)
% \),
% \(
%     l_{n,1} > l_{n,2} > \cdots > l_{n,p}
% \),
% and signs chosen so that
% \(
%     \left( \mPhi^\trans \mU_n \right)_{ii} \geq 0
% \)
% for $1 \leq i \leq p$.  Then
% \begin{enumerate}[(i)]
%     \item $l_{n,i} \toas \lambda_i$ for
%         $1 \leq i \leq p$, and $\mU_n \toas \Phi$ .
%     \item 
%         After appropriate cenering and scaling, 
% 	$\{ l_{n,i} \}_{i=1}^p$ and $\mU_n$
%         converge jointly in distribution and their limits
%         are independent.	
% 	For all $1 \leq i \leq p$
%         \[
%             \sqrt{n} \left( l_{n,i} - \lambda_i \right) 
%             \tod 
%             \Normal\left( 0, \, 2 \lambda_i^2 \right),
%         \]
%         and
%         \[
%             \sqrt{n} \left( \mPhi^\trans \mU_n - \mI_p \right) \tod \mF,
%         \]
%         where $\mF$ is a skew-symmetric matrix with elements above the 
% 	diagonal independent of each other and distributed as 
%         \[
%             F_{ij}
%             \sim
%             \Normal \left(
%                 0, \,
%                 \frac{\lambda_i \lambda_j}
%                      {\left( \lambda_i - \lambda_j \right)^2}
%             \right),
%             \quad
%             \text{for all $1 \leq i < j \leq p$.}
%         \]
% \end{enumerate}
% \end{theorem}
% \begin{proof}
% Part (i) is a restatement of Proposition~\ref{P:suff-stat-limits}.  Part (ii) follows from Proposition~\ref{P:sample-cov-limit} and Theorem~\ref{L:eigen-perturb}.
% \end{proof}


% \section{Modern asymptotics}\label{S:multivariate-modern}

% We now present some results about sample covariance matrices when both the sample size, $n$, and the dimensionality, $p$, go to infinity.  Specifically, most of these results suppose that $n \to \infty$, $p \to \infty$, and 
% \(
%     \frac{n}{p} \to \gamma
% \)
% for a fixed constant
% \(
%     \gamma \in \left( 0, \infty \right).
% \)
% There is no widely-accepted name for $\gamma$, but we will sometimes adopt the terminology of Mar\v{c}enko and Pastur \cite{marcenko1967des} and call it the \emph{concentration}.

% Most of the random matrix theory literature concerning sample covariance matrices is focused on eigenvalues.  Given a sequence of sample covariance matrices $\mS_1, \mS_2, \ldots, \mS_n$, with $\mS_n \in \reals^{p\times p}$ and $p = p(n)$ these results generally come in one of two forms.  If we label the eigenvalues of $\mS_n$ as
% \(
%     l_{n,1}, l_{n,2}, \ldots, l_{n,p},
% \)
% with $l_{n,1} \geq l_{n,2} \geq \cdots \geq l_{n,p}$, then we can define
% a random measure
% \begin{equation}
%     F^{S_n} = \frac{1}{p} \sum_{i=1}^p \delta_{l_{n,i}}.
% \end{equation}
% This measure represents a random draw from the set of eigenvalues of $\mS_n$ that puts equal weight on each eigenvalue.  It is called the \emph{spectral measure} of $\mS_n$.  Results about $F^{S_n}$ are generally called results about the ``bulk'' of the spectrum.  

% The second major class of results is concerned with the behavior of the
% extreme eigenvalues $l_{n,1}$ and $l_{n,p}$. Results of this type are called
% ``edge'' results.

% \subsection{The bulk of the spectrum}

% To work in a setting where the dimensionality $p$, grows with the sample size, $n$, we introduce a triangular array of sample vectors.  The sample covariance matrix $\mS_n$ is of dimension $p\times p$ and is formed from row $n$ of a triangular array of independent random vectors,  
% $\vX_{n,1}, \vX_{n,2}, \ldots, \vX_{n,n}$.  Specifically,
% \(
%     \mS_n
%     =
%     \frac{1}{n}
%     \sum_{i=1}^n \vX_{n,i} \vX_{n,i}^\trans.
% \)
% We let $\mX_n$ be the $n \times p$ matrix
% \(
%     \mX_n
%     =
%     \left(
%     \begin{matrix}
%         \vX_{n,1} &
%         \vX_{n,2} &
%         \cdots &
%         \vX_{n,n}
%     \end{matrix}
%     \right)^\trans,
% \)
% so that 
% \(
%     \mS_n
%     =
%     \frac{1}{n}
%     \mX_n^\trans \mX_n.
% \)
% Most asymptotic results about sample covariance matrices are expressed in terms of $\mX_n$ rather than $\mS_n$.  For example, the next theorem about the spectral measure of a large covariance matrix is stated this way.

% \begin{theorem}\label{T:mp-limit}
% Let $\mX_1, \mX_2, \ldots, \mX_n$ be a sequence of random matrices of increasing dimension as $n \to \infty$, so that $\mX_n \in \reals^{n\times p}$ and $p = p(n)$.
% Define $\mS_n = \frac{1}{n} \mX_n^\trans \mX_n$.  If the elements of $\mX_n$ are iid with $\E| X_{n,11} - \E X_{n,11} |^2 = 1$ and $\frac{n}{p} \to \gamma > 0$, then the empirical spectral measure $F^{\mS_n}$ almost surely converges in distribution to a deterministic probability measure.  This measure, denoted
% $\FMP_\gamma$, is called the Mar\v{c}enko-Pastur Law.  For $\gamma \geq 1$ it
% has density
% \begin{equation}
%     \fMP_\gamma(x)
%     =
%     \frac{\gamma}{2 \pi x}
%     \sqrt{(x - a_\gamma)(b_\gamma - x)},
%     \quad
%     a_\gamma \leq x \leq b_\gamma,
% \end{equation}
% where
% \(
%     a_\gamma = \left( 1 - \frac{1}{\sqrt{\gamma}} \right)^2
% \)
% and
% \(
%     b_\gamma = \left( 1 + \frac{1}{\sqrt{\gamma}} \right)^2.
% \)
% When $\gamma < 1$, there is an additional point-mass of $(1 - \gamma)$ at the origin.
% \end{theorem}

% \noindent
% Figure~\ref{F:mp-law} shows the density $\fMP_\gamma(x)$ for different values
% of $\gamma$.  The reason for choosing the name ``concentration'' to refer to $\gamma$ becomes apparent in that for larger values of $\gamma$, $\FMP_\gamma$ becomes more and more concentrated around its mean.

% The limiting behavior of the empirical spectral measure of a sample covariance matrix was originally studied by Mar\v{c}enko and Pastur~\cite{marcenko1967des} in 1967.  Since then, several papers have refined these results, including Grenander and Silverstein~\cite{grenander1977san}, Wachter~\cite{wachter1978slr},
%  Jonsson~\cite{jonsson1982slt}, Yin and Krishnaiah~\cite{yin1983lte},
% Yin~\cite{yin1986lsd}, Silverstein and Bai~\cite{silverstein1995ede}, and
% Silverstein~\cite{silverstein1995sce}.  These papers either proceed via a combinatorial argument involving the moments of the matrix elements, or else they employ a tool called the Stieltjes transform.  Theorem~\ref{T:mp-limit} is a simplified version of Silverstein and Bai's main result, which more generally considers complex-valued random variables and allows the columns of $\mX_n$ to have heterogeneous variances.

% % \begin{figure}
% %     \centering
% %     \includegraphics{mp-law}
% %     \caption{
% %         \captiontitle{The Mar\v{c}enko-Pastur law}
% %         Density, $\fMP_\gamma(x)$, plotted against quantile, $x$,
% %         for concentration $\gamma = 0.25$, $1$, and $4$.  Concentration
% %         is equal to the number of samples per dimension. For $\gamma < 1$,
% %         there is an addition point-mass of size $(1 - \gamma)$ at $x = 0$.
% %     }\label{F:mp-law}
% % \end{figure}

% The meaning of the phrase ``$F^{\mS_n}$ converges almost surely in distribution to $\FMP_\gamma$'' is that for all $x$ which are continuity points of $\FMP_\gamma$,
% \begin{equation}\label{E:spectral-measure-limit-point}
%     F^{\mS_n} \left(
%         x
%     \right)
%     \toas
%     \FMP_\gamma \left(
%         x
%     \right).
% \end{equation}
% Equivalently, Theorem~\ref{T:mp-limit} can be stated as a strong law of large numbers.

% \begin{corollary}[Wishart LLN]\label{C:wishart-lln}
%     Let $\mX_n$ and $\{ l_{n,i} \}_{i=1}^{p}$ be as in 
%     Theorem~\ref{T:mp-limit}.  Let $g : \reals \to \reals$ be any
%     continuous bounded function.  Then
%     \begin{equation}\label{E:wishart-functional-limit}
%         \frac{1}{p}
%         \sum_{i=1}^p
%             g \left( l_{n,i} \right)
%         \toas
%         \int
%             g(x)
%             \,
%             d\FMP_\gamma ( x ).
%     \end{equation}
% \end{corollary}

% Concerning convergence rates for the quantities in Theorem~\ref{T:mp-limit},  Bai et al. \cite{bai2003crs} study the total variation distance between $F^{\mS_n}$ and $\FMP_\gamma$.  Under suitable conditions on $X_{n,11}$ and $\gamma$, they show that
% \(
%     \| F^{\mS_n} - \FMP_\gamma \|_\text{TV}
%     =
%     \OhP\left( n^{-2/5} \right)
% \)
% and that with probability one
% \(
%     \| F^{\mS_n} - \FMP_\gamma \|_\text{TV}
%     =
%     \Oh\left( n^{-2/5 + \varepsilon} \right)
% \)
% for any $\varepsilon > 0$.
% Guionnet and Zeitouni \cite{guionnet2000csm} give concentration of measure results.  If $X_{n,11}$ satisfies a Poincar\'e inequality and $g$ is Lipschitz, they show that for $\delta$ large enough,
% \begin{equation}\label{E:wishart-ldp}
%     -
%     \frac{1}{n^2}
%     \log \Prob\left\{ 
%         \left|
%             \int g(x) \, dF^{\mS_n}(x) - \int g(x) \, d\FMP_\gamma(x)
%         \right|
%         >
%         \delta
%     \right\}
%     =
%     \Oh\left(
%         \delta^2
%     \right),
% \end{equation}
% with an explicit bound on the error. If one is willing to assume that the elements of $\mX_n$ are Gaussian, then Hiai and Petz~\cite{hiai1998edw} give an exact value for the quantity in \eqref{E:wishart-ldp}.  Guionnet~\cite{guionnetlds} gives a survey of other large deviations results.  

% It is interesting to look at the scaled behavior in equation~\eqref{E:wishart-functional-limit} when the quantities are scaled by $p$.  Indeed one can prove a Central Limit Theorem (CLT) for functionals of eigenvalues.

% \begin{theorem}[Wishart CLT]\label{T:wishart-clt}
%     Let $\mX_1, \mX_2, \ldots, \mX_n$ be a sequence of random $n\times p$ 
%     matrices with $p = p(n)$.  Assume that $\mX_n$ has iid elements and that 
%     $\E\left[ X_{n,11} \right] = 0$, $\E\left[ X_{n,11}^2 \right] = 1$,
%     and $\E\left[ X_{n,11}^4 \right] < \infty$.  
%     Define $\mS_n = \frac{1}{n} \mX_n^\trans \mX_n$ and let $F^{\mS_n}$ be its 
%     spectral measure.  If $n \to \infty$, $\frac{n}{p} \to \gamma$ and $g_1, 
%     g_2, \ldots, g_k$ 
%     are real-valued functions analytic on the support of $\FMP_\gamma$, then 
%     the sequence of random vectors
%     \begin{multline*}
%         p
%         \cdot
%         \Big(
% 	    \int g_1(x) \, dF^{\mS_n}(x)
% 	    -
%             \int g_1(x) \, d\FMP(x), \\
% 	    \int g_2(x) \, dF^{\mS_n}(x)
% 	    -
% 	    \int g_2(x) \, d\FMP(x),
% 	    \ldots, \\
%             \int g_k(x) \, dF^{\mS_n}(x)
% 	    -
%             \int g_k(x) \, d\FMP(x)
% 	\Big)
%     \end{multline*}
%     is tight.
%     Moreover, if $\E \left[ X_{n,11}^4 \right] = 3$, then the sequence 
%     converges in distribution to a multivariate normal with mean $\vmu$ and 
%     covariance $\mSigma$, where 
%     \begin{equation}\label{E:wishart-clt-mean}
%         \mu_i
% 	    =
% 	    \frac{ g_i(a_\gamma) + g_i(b_\gamma) }
%                  { 4 }
% 	    -
%             \frac{1}{2 \pi}
% 	    \int_{a_\gamma}^{b_\gamma}
% 	        \frac{ g_i(x) }
% 	             { \sqrt{(x-a_\gamma)(b_\gamma-x)} }
%                 \,
%                 dx
%     \end{equation}
%     and
%     \begin{equation}\label{E:wishart-clt-cov}
%     	\Sigma_{ij}
%     	=
%     	-
%     	\frac{1}{2 \pi^2}
%     	\iint
%     	    \frac{g_i(z_1) g_j(z_2)}
%     	         {\left( m(z_1) - m(z_2) \right)^2}
%     	    \frac{d}{dz_1} m(z_1)
%     	    \frac{d}{dz_2} m(z_2)
%     	    dz_1 dz_2.
%     \end{equation}
%     The integrals in~\eqref{E:wishart-clt-cov} are contour integrals enclosing the 
%     support of $\FMP_\gamma$, and
%     \begin{equation}
%         m(z)
%         =
% 	    \frac{-(z + 1 - \gamma^{-1}) + \sqrt{(z-a_\gamma)(z-b_\gamma)}}{2 z},
%     \end{equation}
%     with the square root defined to have positive imaginary part when
%     $\Im z > 0$.
% \end{theorem}

% \noindent
% The case when $\E\left[ X_{n,11}^4 \right] = 3$ is of particular interest
% because it arises when $X_{n,11} \sim \Normal\left( 0, 1 \right)$.

% This theorem was proved by Bai and Silverstein~\cite{bai2004clt}, and can
% be considered a generalization of the work by Johnsson~\cite{jonsson1982slt}.
% In computing the variance integral~\eqref{E:wishart-clt-cov}, it is useful to know that $m$ satisfies
% the identities 
% \[
%     m(\bar z) = \overline{m (z) },
% \]
% and
% \[
%     z = - \frac{1}{m(z)} + \frac{\gamma^{-1}}{m(z) + 1}.
% \]
% Bai and Silverstein show how to compute the limiting means and variances
% for $g(x) = \log x$ and $g(x) = x^r$.  They also derive a similar CLT when
% the elements of $\mX_n$ are correlated.  Pastur and Lytova~\cite{pastur2008clt} have recently relaxed some of the assumptions made by Bai and Silverstein.


% \subsection{The edges of the spectrum}

% We now turn our attention to the extreme eigenvalues of a sample covariance
% matrix.  It seems plausible that if $F^{\mS_n} \tod \FMP_\gamma$, then the
% extreme eigenvalues of $\mS_n$ should converge to the edges of the support
% of $\FMP_\gamma$.  Indeed, under suitable assumptions, this is exactly what
% happens.  For the largest eigenvalue, work on this problem started with Geman~\cite{geman1980ltn}, and his assumptions were further weakened by Jonsson~\cite{jonsson1983ole} and Silverstein~\cite{silverstein1984ole}.
% Yin et al.~\cite{yin1988lle} prove the result under the weakest possible
% conditions~\cite{bai1988nle}.

% \begin{theorem}\label{T:max-wishart-eig-limit}
%     Let $\mX_1, \mX_2, \ldots, \mX_n$ be a sequence of random matrices of
%     increasing dimension, with $\mX_n$ of size $n \times p$, $p = p(n)$, 
%     $n \to \infty$, and
%     $\frac{n}{p} \to \gamma \in (0,\infty)$.  Let $\mS_n = \frac{1}{n} \mX_n^\trans \mX_n$ and 
%     denote its eigenvalues by 
%     $l_{n,1} \geq l_{n,2} \geq \cdots \geq l_{n,p}$.  If the 
%     elements of $\mX_n$ are 
%     iid with $\E X_{n,11} = 0$, $\E X_{n,11}^2 = 1$, and
%     $\E X_{n,11}^4 < \infty$, then
%     \[
%         l_{n,1} \toas \left( 1 + \gamma^{-1/2} \right)^2.
%     \]
% \end{theorem}

% For the smallest eigenvalue, the first work was by Silverstein~\cite{silverstein1985sel}, who gives a result when 
% $X_{n,11} \sim \Normal\left( 0,\, 1 \right)$.  Bai and Yin~\cite{bai1993lse} proved a theorem that mirrors Theorem~\ref{T:max-wishart-eig-limit}.

% \begin{theorem}\label{T:min-wishart-eig-limit}
%     Let $\mX_n$, $n$, $p$, and $\{ l_{n,i} \}_{i=1}^p$ be as in
%     Theorem~\ref{T:max-wishart-eig-limit}.  If $\E X_{n,11}^4 < \infty$ and
%     $\gamma \geq 1$, then
%     \[
%         l_{n,p} \toas \left( 1 - \gamma^{-1/2} \right)^2.
%     \]
%     With the same moment assumption on $X_{n,11}$, if $0 < \gamma < 1$, then
%     \[
%         l_{n,p-n+1} \toas \left( 1 - \gamma^{-1/2} \right)^2.
%     \]
% \end{theorem}

% \noindent
% For the case when the elements of $\mX_n$ are correlated, Bai and Silverstein~\cite{bai1998neo} give a general result that subsumes Theorems~\ref{T:max-wishart-eig-limit}~and~\ref{T:min-wishart-eig-limit}.

% After appropriate centering and scaling, the largest eigenvalue of a white
% Wishart matrix converges weakly to a random variable with known distribution.
% Johansson~\cite{johansson2000sfa} proved this statement and identified the
% limiting distribution for complex white Wishart matrices.  Johnstone~\cite{johnstone2001dle} later provided an analogous result for real matrices, which we state below.  

% \begin{theorem}\label{T:tw-limit-largest}
%     Let $\mX_1, \mX_2, \ldots, \mX_n$ be a sequence of random matrices of
%     increasing dimension, with $\mX_n \in \reals^{n\times p}$, $p = p(n)$,
%     and $n \to \infty$ with $\frac{n}{p} \to \gamma \in (0, \infty)$.  Define
%     $\mS_n = \frac{1}{n} \mX_n^\trans \mX_n$ and label its eigenvalues
%     \(
%         l_{n,1} \geq l_{n,2} \geq \cdots \geq l_{n,p}.
%     \)
%     If the elements of $\mX_n$ are iid with
%     \(
%         X_{n,11} \sim \Normal\left( 0, \, 1 \right),
%     \)
%     then
%     \[
%         \frac{l_{n,1} - \mu_{n,p}}{\sigma_{n,p}}
%         \tod
%         W_1
%         \sim
%         \FTW_1,
%     \]
%     where
%     \begin{align*}
%         \mu_{n,p} 
%             &=
%             \frac{1}{n}
%             \left(
%                 \sqrt{n - 1/2}
%                 +
%                 \sqrt{p - 1/2}
%             \right), \\
%         \sigma_{n,p}
%             &= 
%             \frac{1}{n}
%             \left(
%                 \sqrt{n - 1/2}
%                 +
%                 \sqrt{p - 1/2}
%             \right)
%             \left(
%                 \frac{1}{\sqrt{n - 1/2}}
%                 +
%                 \frac{1}{\sqrt{p - 1/2}}
%             \right)^{1/3},
%     \end{align*}
%     and $\FTW_1$ is the Tracy-Widom law of order 1.
% \end{theorem}

% \noindent
% El Karoui~\cite{elkaroui2003lew} extended this result to apply when $\gamma = 0$ or $\gamma = \infty$.  With appropriate modifications to $\mu_{n,p}$ and $\sigma_{n,p}$, he later gave a convergence rate of order $(n \wedge p)^{2/3}$  for complex-valued data \cite{elkaroui2006mpt}.  Ma~\cite{ma2008atw} gave the analogous result for real-valued data.  For correlated complex normals, El Karoui~\cite{elkaroui2007twl} derived a more general version of Theorem~\ref{T:tw-limit-largest}.  

% The Tracy-Widom distribution, which appears in Theorem~\ref{T:tw-limit-largest}, was  established to be the limiting distribution (after appropriate scaling) of the maximum eigenvalue from an $n \times n$ symmetric matrix with independent entries distributed as $\Normal\left( 0, \, 2 \right)$ along the main diagonal and $\Normal\left( 0, \, 1 \right)$ otherwise \cite{tracy1994lsd} \cite{tracy1996oas}.  To describte $\FTW_1$, let $q(x)$ solve the Painlev\'e II equation
% \[
%     q''(x) = x q(x) + 2 q^3(x),
% \]
% with boundary condition $q(x) \sim \Ai(x)$ as $x \to \infty$ and $\Ai(x)$ the Airy function.  Then it follows that
% \[
%     \FTW_1(x)
%     =
%     \exp \left\{
%         -
%         \frac{1}{2}
%         \int_s^\infty
%             q(x)
%             +
%             (x-s) q^2 (x)
%             \,
%             dx
%     \right\}.
% \]
% Hastings and McLeod~\cite{hastings1980bvp} study the tail behavior of $q(x)$.  Using their analysis, one can show (see, e.g. \cite{perry2009mre}) that for $s \to -\infty$, 
% \[
%     \FTW_1(s)
%     \sim
%     \exp \left(
%         -\frac{|s|^3}{24}
%     \right),
% \]
% while for $s \to \infty$,
% \[
%     1 - \FTW_1(s)
%     \sim
%     \frac{ s^{-3/4} }
%          { 4 \sqrt{ \pi } }
%     \exp \left(
%         -
%         \frac{2}{3}
%         s^{3/2}
%     \right).
% \]
% The density of $\FTW_1$ is shown in Figure~\ref{F:tw-density}.

% % \begin{figure}
% %     \centering
% %     \includegraphics{tw-law}
% %     \caption{
% %         \captiontitle{The Tracy-Widom law}
% %         Limiting density of the largest eigenvalue from a white Wishart
% %         matrix after appropriate centering and scaling.
% %     }
% %     \label{F:tw-density}
% % \end{figure}

% A result like Theorem~\ref{T:tw-limit-largest} holds true for the smallest eigenvalue.  We define the \emph{Reflected Tracy-Widom Law} to have distribution function $\GTW_1(s) = 1 - \FTW_1(-s)$.  Then we have 

% \begin{theorem}\label{T:tw-limit-smallest}
%     With the same assumptions as in Theorem~\ref{T:tw-limit-largest}, if
%     $\gamma \in (1,\infty)$ then
%     \[
%         \frac{l_{n,p} - \mu_{n,p}^-}{\sigma_{n,p}^-}
%         \tod
%         W_1
%         \sim
%         \GTW_1,
%     \]
%     where
%     \begin{align*}
%         \mu_{n,p}^-
%             &=
%             \frac{1}{n}
%             \left(
%                 \sqrt{n - 1/2}
%                 -
%                 \sqrt{p - 1/2}
%             \right), \\
%         \sigma_{n,p}^-
%             &= 
%             \frac{1}{n}
%             \left(
%                 \sqrt{n - 1/2}
%                 -
%                 \sqrt{p - 1/2}
%             \right)
%             \left(
%                 \frac{1}{\sqrt{p - 1/2}}
%                 -
%                 \frac{1}{\sqrt{n - 1/2}}
%             \right)^{1/3}.
%     \end{align*}
%     If $\gamma \in (0, 1)$, then
%     \[
%         \frac{l_{n,p-n+1} - \mu_{p,n}^-}{\sigma_{p,n}^-}
%         \tod
%         W_1
%         \sim
%         \GTW_1.
%     \]
% \end{theorem}

% \noindent
% We get the result for $\gamma \in (0,1)$ by reversing the role of $n$ and $p$.
% Baker et al~\cite{baker1998rme} proved the result for complex data.
% Paul~\cite{paul2006dse} extended the result to real data when $\gamma \to \infty$.  Ma~\cite{ma2008atw} gives convergence rates.  In practice,
% $\log l_{n,p}$ converges in distribution faster than $l_{n,p}$.  Ma recommends appropriate centering and scaling constants for $\log l_{n,p}$ to converge in distribution to a $\GTW$ random variable at rate $(n \wedge p)^{2/3}$.

% Theorem~\ref{T:tw-limit-smallest} does not apply when $\frac{n}{p} \to 1$.  Edelman~\cite{edelman1988ecn} derived the limiting distribution of the smallest eigenvalue when $n = p$.  It is not known if his result holds more generally when $\frac{n}{p} \to 1$.

% \begin{theorem}\label{T:edelman-limit-smallest}
%     Let $\mX_n$ and $\{ l_{n,i} \}_{i=1}^p$ be as in 
%     Theorem~\ref{T:tw-limit-largest}.  If $p(n) = n$, then for $t \geq 0$,
%     \[
%         \Prob\left\{
%             n \, l_{n,p} \leq t
%         \right\}
%         \to
%         \int_0^t
%             \frac{1 + \sqrt{x}}{\sqrt{x}}
%             e^{-(x/2 + \sqrt{x})}
%             \,
%             dx.
%     \]
% \end{theorem}

% In addition to the extreme eigenvalues, it is possible to study the joint distribution of top or bottom $k$ sample eigenvalues for fixed $k$ as
% $n \to \infty$.  In light of Theorem~\ref{T:mp-limit}, for fixed $k$ we must have that the top (respectively, bottom) sample eigenvalues converge almost surely to the same limit.  Furthermore, Soshnikov~\cite{soshnikov2002nud} showed that applying the centering and scaling from Theorem~\ref{T:tw-limit-largest} to the top $k$ sample eigenvalues gives a specific limiting distribution.

% It is natural to ask if the limiting eigenvalue distributions are specific to Wishart matrices, or if they apply to non-Gaussian data as well.  There is compelling evidence that the Tracy-Widom law is universal.   Soshnikov~\cite{soshnikov2002nud} extended Theorem~\ref{T:tw-limit-largest} to more general $\mX_n$ under the assumption that $X_{n,11}$ is sub-Gaussian and $n-p = \Oh(p^{1/3})$.  P\'ech\'e~\cite{peche2008url} later removed the restriction on $n-p$.  Tao and Vu~\cite{tao2009rmd} showed that Theorem~\ref{T:edelman-limit-smallest} applies for general $X_{n,11}$ with $\E X_{n,11} = 0$ and $\E X_{n,11}^2 = 1$.


% \subsection{Eigenvectors}

% Relatively little attention has been focused on the eigenvectors of sample covariance matrices.  While many results are known, as of yet there is no complete characterization of the eigenvectors from a general sample covariance matrix.  Most of the difficulty in tackling the problem is that it is hard to describe convergence properties of $\mU_n$, the $p\times p$ matrix of eigenvectors, when $n$ and $p$ go to infinity.  The individual $p^2$ elements of $\mU_n$ do not converge in any meaningful way, so the challenge is to come up with relevant macroscopic characteristics of $\mU_n$.

% Silverstein~\cite{silverstein1979reg} was perhaps the first to study the eigenvectors of large-dimensional sample covariance matrices.  He hypothesized that for sample covariance matrices of increasing dimension, the eigenvector matrix becomes more and more ``Haar-like''.  A random matrix $\mU \in \reals^{p\times p}$ is said to be Haar-distributed over the orthogonal group if for every fixed $p \times p$ orthogonal matrix $\mO$, the rotated matrix $\mO \mU$ has the same distribution as $\mU$.  That is, $\mO \mU \eqd \mU$.  Silverstein's conjecture was that as $n \to \infty$, $\mU_n$ behaves more and more like a Haar-distributed matrix.  The next theorem displays one aspect of Haar-like behavior.

% To state the theorem, we need to define the extension of a scalar function
% $g : \reals \mapsto \reals$ to symmetric matrix arguments.  If 
% $\mS = \mU \mL \mU^\trans$ is the eigendecomposition of the symmetric
% matrix $\mS \in \reals^{p \times p}$, with 
% $\mL = \diag(l_1, l_2, \ldots, l_p)$, then we define
% \[
%     g\left( \mS \right)
%         =
%             \mU
%             \Big(
%                 \diag\big(
%                     g(l_1), g(l_2), \ldots, g(l_p)
%                 \big)
%             \Big)
%             \mU^\trans.
% \]
% With this notion, we can state Silverstein's result.

% \begin{theorem}\label{T:evec-functional-limit}
%     Let $\mX_1, \mX_2, \ldots, \mX_n$ be a sequence of random matrices of
%     increasing dimension, with $\mX_n \in \reals^{n \times p}$, $p = p(n)$,
%     and $\mX_n$ having iid elements with 
%     \(
%         \E X_{n,11} = 0,
%     \)
%     \(
%         \E X_{n,11}^2 = 1,
%     \)
%     and
%     \(
%         \E X_{n,11}^4 < \infty.
%     \)
%     Define $\mS_n = \frac{1}{n} \mX_n^\trans \mX_n$.  Let
%     $\va_1, \va_2, \ldots, \va_n$ be a sequence of nonrandom unit vectors with 
%     $\va_n$ in $\reals^p$ and let $g : \reals \to \reals$ be a continuous 
%     bounded function.  If $n \to \infty$ and
%     $\frac{n}{p} \to \gamma \in (0,\infty)$, then
%     \[
%         \va_n^\trans g\left( \mS_n \right) \va_n
%         \toas
%         \int
%             g(x)
%             \,
%             d\FMP_\gamma(x).
%     \]
% \end{theorem}

% \noindent
% Silverstein~\cite{silverstein1979reg} proves the result for convergence in probability and a specific class of $\mX_n$.  Bai et al.~\cite{bai2007ael} strengthen the result to a larger class of $\mX_n$ and proves almost-sure convergence.  They also consider dependence in $\mX_n$.

% It may not be immediately obvious how Theorem~\ref{T:evec-functional-limit}
% is related to eigenvectors.  If $\mS_n = \mU_n \mL_n \mU_n$ is the
% eigendecomposition of $\mS_n$, with 
% \(
%     \mL_n
%     = 
%     \diag\left(
%         l_{n,1}, l_{n,2}, \ldots, l_{n,p}
%     \right),
% \)
% then
% \(
%     g(\mL_n)
%     =
%     \diag \big(
%         g(l_{n,1}), g(l_{n,2}), \ldots, g(l_{n,p})
%     \big)
% \)
% and
% \(
%     g(\mS_n)
%     =
%     \mU_n
%     g(\mL_n)
%     \mU_n^\trans.
% \)
% We let $\vb_n = \mU_n \va_n$.  Then,
% \begin{equation}\label{E:evector-functional}
%     \va_n^\trans g(\mS_n) \va_n
%     =
%     \sum_{i=1}^p
%         b_{n,i}^2
%         \,
%         g( l_{n,i} ).
% \end{equation}
% If $\mU_n$ is Haar-distributed, then $\vb_n$ will be distributed uniformly over the unit sphere in $\reals^p$, and the average in \eqref{E:evector-functional} will put about weight $\frac{1}{p}$ on each eigenvalue.  If $\mU_n$ puts bias in any particular direction then the average will put extra weight on particular eigenvalues.

% Silverstein investigated second-order behavior of eigenvectors in \cite{silverstein1981dbe}, \cite{silverstein1984slt}, \cite{silverstein1989eld}, and \cite{silverstein1990wcr}.  He demonstrated that certain second-order behavior of $\mU_n$ depends in a crucial way on the fourth moment of $X_{n,11}$.  This greatly restricts the class of $\mX_n$ for with the eigenvectors of $\mS_n$ are Haar-like.

% \begin{theorem}\label{T:evec-functional-scaled-limit}
%     Let $\mX_n$, $\mS_n$, and $\va_n$ be as in 
%     Theorem~\ref{T:evec-functional-limit}. Suppose also that
%     $\E X_{n,11}^4 = 3$.  Let $g_1, g_2, \ldots, g_k$
%     be real-valued functions analytic on the support of $\FMP_\gamma$.  
%     Then, the random vector
%     \begin{multline*}
%         \sqrt{p} \cdot
%         \left(
%             \va_n^\trans g_1( \mS_n ) \va_n
%             -
%             \int g_1(x) \, d\FMP_\gamma(x), \right. \\
%             \va_n^\trans g_2( \mS_n ) \va_n
%             -
%             \int g_2(x) \, d\FMP_\gamma(x),
%             \ldots, \\ \left.
%             \va_n^\trans g_k( \mS_n ) \va_n
%             -
%             \int g_k(x) \, d\FMP_\gamma(x)
%         \right)
%     \end{multline*}
%     converges in distribution to a mean-zero multivariate normal with
%     covariance between the $i$th and $j$th components equal to
%     \[
%         \int
%             g_i(x) g_j(x) \, d\FMP_\gamma(x)
%         -
%         \int
%             g_i(x) \, d\FMP_\gamma(x)
%         \cdot
%         \int
%             g_j(x) \, d\FMP_\gamma(x).
%     \]
% \end{theorem}

% \noindent
% Bai et al.~\cite{bai2007ael} give a similar result for complex-valued and correlated $\mX_n$.  Silverstein~\cite{silverstein1989eld} showed that if $g_1(x) = x$ and $g_2(x) = x^2$, then the condition $\E X_{n,11}^4 = 3$ is necessary for the random vector in Theorem~\ref{T:evec-functional-scaled-limit} to converge in distribution for all $\va_n$.  However, for the specific choice of
% \(
%     \va_n
%     =
%     \left(
%         \frac{1}{\sqrt{p}},
%         \frac{1}{\sqrt{p}},
%         \ldots,        
%         \frac{1}{\sqrt{p}}        
%     \right),
% \)
% he later showed that the conclusions of Theorem~\ref{T:evec-functional-scaled-limit} hold more generally when $X_{n,11}$ is symmetric and $\E X_{n,11}^4 < \infty$ \cite{silverstein1990wcr}.  
