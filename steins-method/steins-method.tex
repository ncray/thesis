\chapter{Stein's method}\label{C:steins-method}
In this chapter we present an introduction to Stein's method of exchangeable pairs which we use to
prove the core theoretical result of this thesis: a rate of convergence bound for the randomization
distribution.

\section{Introduction}\label{S:steins-method-introduction}
Stein's method provides a means of bounding the distance between two probability distributions in a
given probability metric.  When applied with the normal distribution as the target, this results in
central limit type theorems.  Several flavors of Stein's method (e.g. the method of exchangeable
pairs) proceed via auxiliary randomization.  We reproduce Stein's proof of the Hoeffding
combinatorial central limit theorem (HCCLT) with explicit calculation of various constants.  It will
be instructive to follow the proof of the HCCLT because our proof proceeds in a similar fashion but
with the following generalizations: an approximate contraction property, less cancellation of terms
due to separate estimation of various denominators, and non-unit variance of an r.v. in the
exchangeable pair.

\section{Hoeffding combinatorial CLT}
\begin{theorem}
  Let $\{a_{ij}\}_{i,j}$ be an $n \times n$ matrix of real-valued entries that is row- and
  column-centered and scaled such that the sums of the squares of its elements equals $n-1$:
  \begin{align}
    \sum_{j=1}^n a_{ij} &= 0 \\
    \sum_{i=1}^n a_{ij} &= 0 \\
    \sum_{i=1, j=1}^n a_{ij}^2 &= n-1
  \end{align}
  Let $\Pi$ be a random permutation of $\{1, \ldots, n\}$ drawn uniformly at random from the set of
  all permutations:
  \begin{equation}
    P(\Pi = \pi) = \frac{1}{n!}.
  \end{equation}
  Define
  \begin{equation}
    W = \sum_{i=1}^n a_{i\Pi(i)}
  \end{equation}
  to be the sum of a random diagonal.  Then
  \begin{equation}
    |P(W \leq w) - \Phi(w)| \leq 
    \frac{C}{\sqrt{n}} \left [
      \sqrt{\sum_{i, j = 1}^n a_{ij}^4} + 
      \sqrt{\sum_{i, j = 1}^n |a_{ij}|^3} 
    \right ].
  \end{equation}
  \begin{proof}
    In order to construct our exchangeable pair, we introduce the ordered pair of random variables
    $(I, J)$ independent of $\Pi$ that represents a uniformly at random draw from the set of all
    non-null transpositions:
    \begin{equation}
      P(I = i, J = j) = \frac{1}{n(n-1)} \quad i, j \in \{1, \ldots, n\}, i \neq j.
    \end{equation}
    Define the random permutation $\Pi'$ by
    \begin{equation}
      \Pi'(i) = \Pi \circ (I, J) = 
      \begin{cases}
        \Pi(J) \quad i = I \\
        \Pi(I) \quad i = J \\
        \Pi(i) \quad \text{else}.
      \end{cases}
    \end{equation}
    We construct our exchangeable pair by defining
    \begin{equation}
      W' = \sum_{i=1}^n a_{i\Pi'(i)} = W 
      - a_{I \Pi(I)} + a_{I \Pi(J)} - a_{J \Pi(J)} + a_{J \Pi(I)}.
    \end{equation}
    We now verify the contraction property:
    \begin{align}
      \E[W - W' | \Pi] &=
      \E[a_{I \Pi(I)} - a_{I \Pi(J)} + a_{J \Pi(J)} - a_{J \Pi(I)} | \Pi] \\
      & = \frac{2}{n}\sum_{i=1}^n a_{i\Pi(i)} - 
      \frac{2}{n}\frac{1}{n-1}\sum_{i,j=1, i \neq j}^n a_{i\Pi(j)} \\
      &= \frac{2}{n} W - 
      \frac{2}{n}\frac{1}{n-1} 
      \left [ 
        \sum_{i,j=1}^n a_{i\Pi(j)} - \sum_{i}^n a_{i\Pi(i)}
      \right ] \\
      &= \frac{2}{n} W + \frac{2}{n}\frac{1}{n-1}W - 
      \frac{2}{n}\frac{1}{n-1}
      \left [ 
        \sum_{i=1}^n \sum_{j=1}^n a_{i\Pi(j)}
      \right ] \\
      &= \frac{2}{n} W \left (1 + \frac{1}{n-1} \right ) - 0 \\
      &= \frac{2}{n-1} W
    \end{align}
    This satisfies our contraction property with 
    \begin{equation}
      \lambda = \frac{2}{n-1}.
    \end{equation}
    
    To bound the variance component, compute
    \begin{equation}
      \begin{split}
        \E [(W-W')^2 | \Pi] 
        &= \E[(a_{I \Pi(I)} - a_{I \Pi(J)} + a_{J \Pi(J)} - a_{J \Pi(I)})^2 | \Pi] \\
        &= \E[ a_{I \Pi(I)}^2 + a_{J \Pi(J)}^2 + a_{I \Pi(J)}^2 + a_{J \Pi(I)}^2 \\
        &\quad - 2a_{I \Pi(I)} a_{I \Pi(J)} - 2a_{J \Pi(J)} a_{J \Pi(I)}
        - 2a_{I \Pi(I)} a_{J \Pi(I)} - 2a_{J \Pi(J)} a_{I \Pi(J)} \\
        &\quad + 2a_{I \Pi(I)} a_{J \Pi(J)} + 2a_{I \Pi(J)} a_{J \Pi(I)} | \Pi ] \\
        &= \frac{2}{n} \sum_{i=1}^n a_{i \Pi(i)}^2 
        + \frac{2}{n}\frac{1}{n-1}\sum_{i,j=1, i\neq j}^n a_{i \Pi(j)}^2 \\
        &\quad - \frac{4}{n}\frac{1}{n-1}\sum_{i,j=1, i\neq j}^n a_{i \Pi(i)}a_{i \Pi(j)}
        - \frac{4}{n}\frac{1}{n-1}\sum_{i,j=1, i\neq j}^n a_{i \Pi(i)}a_{j \Pi(i)} \\
        &\quad + \frac{2}{n}\frac{1}{n-1}\sum_{i,j=1, i\neq j}^n a_{i \Pi(i)}a_{j \Pi(j)}
        + \frac{2}{n}\frac{1}{n-1}\sum_{i,j=1, i\neq j}^n a_{i \Pi(j)}a_{j \Pi(i)} \\
        &= \frac{2}{n} \sum_{i=1}^n a_{i \Pi(i)}^2 
        + \frac{2}{n}\frac{1}{n-1} \left ( 
          \sum_{i,j=1}^n a_{i \Pi(j)}^2 - \sum_{i=1}^n a_{i \Pi(i)}^2 \right ) \\
        &\quad - \frac{4}{n}\frac{1}{n-1} \sum_{i=1}^n \left ( a_{i\Pi(i)} 
        \sum_{j=1}^n \left ( a_{i\Pi(j)} + a_{j\Pi(i)} \right ) - 2 a_{i \Pi(i)}^2 
        \right ) \\
        &\quad + \frac{2}{n}\frac{1}{n-1} \left ( \sum_{i=1}^n \sum_{j=1}^n \left ( 
          a_{i\Pi(i)}a_{j\Pi(j)} + a_{i\Pi(j)}a_{j\Pi(i)}
          \right )  - 2 \sum_{i=1}^n a_{i \Pi(i)}^2 \right ) \\
        &= \frac{2}{n}\left ( 1 - \frac{1}{n-1} \right )  \sum_{i=1}^n a_{i \Pi(i)}^2 
        + \frac{2}{n} \\
        &\quad + \frac{8}{n}\frac{1}{n-1} \sum_{i=1}^n a_{i \Pi(i)}^2 \\
        &\quad + \frac{2}{n}\frac{1}{n-1} \sum_{i=1}^n \sum_{j=1}^n \left ( 
          a_{i\Pi(i)}a_{j\Pi(j)} + a_{i\Pi(j)}a_{j\Pi(i)}
        \right ) - \frac{4}{n}\frac{1}{n-1} \sum_{i=1}^n a_{i \Pi(i)}^2 \\
        &= \frac{2}{n} + \frac{2}{n-1}\sum_{i=1}^n a_{i \Pi(i)}^2 + \frac{2}{n}\frac{1}{n-1}
        \sum_{i=1}^n \sum_{j=1}^n (a_{i\Pi(i)}a_{j\Pi(j)} + a_{i\Pi(j)}a_{j\Pi(i)})
      \end{split}
    \end{equation}

    \begin{theorem}[The $c_r$-inequality]
      \label{T:c_r-inequality}
      Let $r > 0$.  Suppose that $\E|X|^r < \infty$ and $\E|Y|^r < \infty$.  Then
      \begin{equation}
        \E|X + Y|^r < c_r(\E|X|^r+\E|Y|^r),
      \end{equation}
      where $c_r = 1$ when $r \leq 1$ and $c_r = 2^{r-1}$ when $r \geq 1$.
    \end{theorem}

    \begin{corollary}
      \label{C:sum_variance}
      Suppose that $\var(X) < \infty$ and $\var(Y) < \infty$.  Then
      \begin{equation}
        \var(X + Y) < 2(\var(X)+\var(Y)).
      \end{equation}
    \end{corollary}
    \begin{proof}
      This follows immediately from applying Theorem~\ref{T:c_r-inequality} to the centered random
      variables $X' = X - \E[X]$ and $Y' = Y - \E[Y]$.
    \end{proof}
    
    

  \end{proof}
\end{theorem}