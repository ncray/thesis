\chapter{Stein's method}
\label{C:steins-method}
In this chapter, we present an introduction to
Stein's method of exchangeable pairs, which we use to prove the core
theoretical result of this thesis: a rate of convergence bound on the
Kolmogorov distance between the randomization distribution of the $t$-statistic
and the standard normal distribution.

Due to similarities between this problem and Hoeffding's combinatorial central
limit theorem (HCCLT), we first review the HCCLT and related results.

\section{Introduction}
\label{S:steins-method-introduction}
Stein's method provides a means of bounding the distance between two
probability distributions in a given probability metric.  When applied
with the normal distribution as the target, this results in
central-limit-type theorems.  Traditional results of this type
typically make use of the complex-valued characteristic function.  For
an account of the early history and many variations of the Central
Limit Theorem, see Le Cam \cite{cam1986central}.  In
\cite{stein1972bound}, Stein introduced a groundbreaking approach with
a real characterizing equation in order to more effectively tame
various forms of dependence.

He soon refined these initial ideas into his monograph
\cite{stein1986approximate}, which includes the Poisson approximation
work of Chen \cite{chen1975poisson}, random graph results of Barbour
and Eagleson \cite{barbour1985multiple}, and binomial approximation of
Diaconis \cite{diaconis1977distribution}.  Several flavors of Stein's
method (e.g., the method of exchangeable pairs) proceed via auxiliary
randomization.  Other compilations with many illuminating applications
include those of Diaconis and Holmes \cite{diaconis2004stein}, Barbour
and Chen \cite{barbour2005introduction, barbour2005stein}, and
Chen, Goldstein, and Shao \cite{chen2010normal}.

It will be instructive to follow the proof of the HCCLT in \cite{stein1986approximate}
because our proof proceeds in a similar fashion but with the following
generalizations: an approximate contraction property, less
cancellation of terms due to separate estimation of various
denominators, and non-unit variance of the random variable of interest.
In Appendix~\ref{A:steins-method-app}, we prove various
generalizations of theorems in \cite{chen2010normal}.
We also refer to auxiliary results in Appendix~\ref{A:auxiliary-app}.

\section{Stein's Theorem}
Theorem~\ref{T:stein-theorem} bounds the Kolmogorov distance between the
distribution of the random variable $W$ and the standard normal distribution
in terms of functions of the difference of the exchangeable pair $(W, W')$.  It is
applied to the situation where $W$ is the sum of the random diagonal of a matrix
to prove the HCCLT.  Chen et al.\ \cite{chen2010normal}
generalize Theorem~\ref{T:stein-theorem} to allow
for situations in which the regression condition does not hold exactly, and
we later in addition relax the assumption that $W$ has unit variance.
\begin{theorem}[Stein]\label{T:stein-theorem}
  If $W$, $W'$ are mean 0, exchangeable random variables with variance 1
  satisfying the exact regression condition
  \begin{equation*}
    \E[W'-W|W] = -\lambda W
  \end{equation*}
  for some $\lambda \in (0,1)$, then
  \begin{align*}
    \sup_{w \in \mathbb{R}} |P(W \leq w) - \Phi(w)| &\leq
    2\sqrt{\E \left [1 - \frac{1}{2 \lambda}E[(W'-W)^2|W] \right ]} +
    (2 \pi)^{-1/4}\sqrt{\frac{1}{\lambda}\E |W'-W|^3} \\
    &\leq 2 \sqrt{\var (\E [(W'-W)^2|W])} +
    (2 \pi)^{-1/4}\sqrt{\frac{1}{\lambda}\E |W'-W|^3}.
  \end{align*}
\end{theorem}

\section{Hoeffding's Combinatorial CLT}
Stein's proof of the HCCLT relies on an application of Theorem~\ref{T:stein-theorem}.
\begin{theorem}[Hoeffding's Combinatorial Central Limit Theorem]\label{T:HCCLT}
  Let $\{a_{ij}\}_{i,j}$ be an $n \times n$ matrix of real-valued
  entries that is row- and column-centered and scaled such that the sums
  of the squares of its elements equals $n-1$:
  \begin{align*}
    \sum_{j=1}^n a_{ij} &= 0 \\
    \sum_{i=1}^n a_{ij} &= 0 \\
    \sum_{i=1, j=1}^n a_{ij}^2 &= n-1
  \end{align*}
  Let $\Pi$ be a random permutation of $\{1, \ldots, n\}$ drawn
  uniformly at random from the set of all permutations:
  \begin{equation*}
    P(\Pi = \pi) = \frac{1}{n!}.
  \end{equation*}
  Define
  \begin{equation*}
    W = \sum_{i=1}^n a_{i\Pi(i)}
  \end{equation*}
  to be the sum of a random diagonal.  Then
  \begin{equation*}
    |P(W \leq w) - \Phi(w)| \leq
    \frac{C}{\sqrt{n}} \left [
      \sqrt{\sum_{i, j = 1}^{n} a_{ij}^4} +
      \sqrt{\sum_{i, j = 1}^{n} |a_{ij}|^3}
    \right ].
  \end{equation*}
\end{theorem}

Originally proved by Hoeffding using the method of moments
\cite{hoeffding1951combinatorial}, Theorem~\ref{T:HCCLT} generalizes
the non-parametric measure of rank correlation known as Spearman's
footrule \cite{spearman1904proof} \cite{salama1990note}
\begin{equation*}
  D(\pi, \sigma) = \sum_{i=1}^n |\pi(i) - \sigma(i)|,
\end{equation*}
where $\pi$ and $\sigma$ are elements of $S_n$, the set of
permutations of $n$ letters.  Considering $D$ as a metric on $S_n$,
Diaconis and Graham \cite{diaconis1977spearman} related it to other
common non-parametric measures and derived various asymptotic results.
In fact, Spearman's footrule has found contemporary applications such
as in ranking search queries and microarray experiments
\cite{sens2011spearman}, where consistency of the top $k$ out of $n$
objects is desired.

Theorem~\ref{T:HCCLT} was extended by Motoo \cite{motoo1956hoeffding} and
also generalizes the work on the permutation-based
tests of Wald and Wolfowitz \cite{wald1944statistical} and Noether
\cite{noether1949theorem}.  These results build further on the problem of
devising exact tests of significance when the underlying probability
distribution is unknown, originating from Fisher
\cite{fisher1935design, fisher1970statistical}.

Given fixed data $a_{ij}$, the HCCLT provides a bound in terms of a
universal constant $C$, a sample-size-dependent term
$\frac{1}{\sqrt{n}}$, and a function of the data $\sqrt{\sum_{i, j =
1}^{n} a_{ij}^4} + \sqrt{\sum_{i, j = 1}^{n} |a_{ij}|^3}$.  Thus,
given $C$, we can calculate an explicit bound on the Kolmogorov
distance.

Consider a sequence of matrices, $a_{ij}^{(n)}$, of dimension $n
\times n$.  In order to achieve an $\mathcal{O}(n^{-1/2})$ rate of
convergence, we require that the function of the data be bounded.
However, this will not typically be the case.

Bolthausen \cite{bolthausen1984estimate} proved the following result:
\begin{theorem}[Bolthausen]
  Under the conditions of Theorem~\ref{T:HCCLT}, there is an absolute constant
  $K > 0$ such that
  \begin{equation*}
    |P(W \leq w) - \Phi(w)| \leq K \frac{\sqrt{\sum_{i, j = 1}^{n} |a_{ij}|^3}}{n}.
  \end{equation*}
\end{theorem}

Then, given the sequence of matrices $a_{ij}^{(n)}$, the theorem
yields a convergence rate of $\mathcal{O}(n^{-1/2})$ as long as
$\sqrt{\sum_{i, j = 1}^{n} |a_{ij}|^3} / \sqrt{n}$ remains bounded.

von Bahr \cite{bahr1976remainder} and Ho and Chen \cite{ho1978l_p}
also obtained the same rate under some boundedness conditions
on random-valued entries of the matrix.  For example, von
Bahr obtained the following result:
\begin{theorem}[von Bahr]
  Let $\{X_{ij}\}_{i, j = 1}^n$ be a square matrix of random variables with
  independent row vectors.
  With $\mu(i, j) = \E X_{ij}$, define
  \begin{equation*}
    \bar{\mu}(i, \cdot) = n^{-1} \sum_j \mu(i, j), \quad
    \bar{\mu}(\cdot, j) = n^{-1} \sum_i \mu(i, j), \quad
    \bar{\mu}(\cdot, \cdot) = n^{-2} \sum_i \sum_j \mu(i, j). \quad
  \end{equation*}
  Set $Y_{ij} = X_{ij} - \bar{\mu}(i, \cdot) - \bar{\mu}(\cdot, j) + \bar{\mu}(\cdot, \cdot)$,
  $\tau_{ij}^2 = \E Y_{ij}^2$, and $\tau^2 = n^{-2} \sum_{i, j} \tau_{ij}^2$.

  Let $\Pi$ be a uniformly at random permutation, independent of $X_{ij}$, and
  $W = \sum_{i=1}^n X_{i\Pi(i)}$ be the sum of a random diagonal.
  Then there exists an absolute constant $C$ such that for all $w \in \mathbb{R}$,
  \begin{equation*}
    \left |P \left (\frac{W - \mu_W}{\sigma_W} \leq w \right ) - \Phi(w) \right |
    \leq \frac{C}{\sqrt{n}} \gamma,
  \end{equation*}
  where $\mu_W$ denotes the mean of $W$, $\sigma_W$ its standard deviation,
  $\Phi(w)$ the standard normal CDF, and
  \begin{equation*}
    \gamma = \max \left [
      \max_{i, j} \E |Y_{ij}| / \tau,
      \max_i \sum_j \E Y_{ij}^2 / n \tau^2,
      \max_j \sum_i \E Y_{ij}^2 / n \tau^2,
      \max_{i, j} \sum_{i, j} \E |Y_{ij}|^3 / n^2 \tau^3
    \right ].
  \end{equation*}
\end{theorem}

Now, we return to generalizing Theorem~\ref{T:stein-theorem}.

\section{Generalized Stein's Theorems}
Here, we treat the situation where the regression condition fails to hold exactly.
Chen et al.\ \cite{chen2010normal} serves as an excellent reference for results
of this type.

\begin{definition}[Approximate Stein Pair]
  Let $(W, W')$ be an exchangeable pair.  If the pair satisfies the ``approximate linear
  regression condition''
  \begin{equation}
    \label{D:approx-stein-pair}
    \E [W - W' | W] = \lambda (W - R),
  \end{equation}
  where $R$ is a variable of small order and $\lambda \in (0, 1)$, then we call $(W, W')$ an
  approximate Stein pair.
\end{definition}

Here we generalize Lemma 5.1 from \cite{chen2010normal} to the setting of non-unit variance:
\begin{theorem}
  \label{L:stein-difference-second-moment-generalization-bound}
  If $W$, $W'$ are mean 0 exchangeable random variables with variance $\E W^2$
  satisfying
  \begin{equation*}
    \E[W'-W|W] = -\lambda(W-R)
  \end{equation*}
  for some $\lambda \in (0,1)$ and some random variable $R$, then for any
  $z \in \mathbb{R}$ and $a > 0$,
  \begin{equation*}
    \E [(W'-W)^2 \mathbf{1}_{\{-a \leq W' - W \leq 0\}} \mathbf{1}_{\{z-a \leq W \leq z\}}] \leq
    3a \lambda (\sqrt{\E W^2} + \E |R|)
  \end{equation*}
  and
  \begin{equation*}
    \E [(W'-W)^2 \mathbf{1}_{\{0 \leq W' - W \leq a\}} \mathbf{1}_{\{z-a \leq W \leq z\}}] \leq
        3a \lambda (\sqrt{\E W^2} + \E |R|).
  \end{equation*}
\end{theorem}

The following theorem will allow us to prove an $\mathcal{O}(n^{-1/4})$ rate under
mild conditions.

Here, we generalize Theorem 5.5 from \cite{chen2010normal}:
\begin{theorem}
  \label{T:main}
  If $W$, $W'$ are mean 0 exchangeable random variables with variance
  $\E W^2$
  satisfying
  \begin{equation*}
    \label{eq:13}
    \E[W'-W|W] = -\lambda(W-R)
  \end{equation*}
  for some $\lambda \in (0,1)$ and some random variable $R$, then
  \begin{equation*}
    \begin{split}
      \sup_{z \in \mathbb{R}} |P(W \leq z) - \Phi(z)|
      &\leq (2\pi)^{-1/4} \sqrt{\frac{\E |W'-W|^3}{\lambda}}
      + \frac{1}{2\lambda} \sqrt{\var (\E [(W'-W)^2|W])} \\
      &\quad + |\E W^2 - 1| + \E |WR| + \E |R|
    \end{split}
  \end{equation*}
\end{theorem}

The following result will let us achieve a rate of $\mathcal{O}(n^{-1/2})$ subject to
the difference $|W'-W|$ being bounded.  This condition has been similarly used by
Rinott and Rotar \cite{rinott1997coupling} and Shao and Su \cite{shao2006berry}.
It generalizes part of Theorem 5.3 from \cite{chen2010normal}:
\begin{theorem}
  \label{T:better-rate}
  If $W$, $W'$ are mean 0 exchangeable random variables with variance
  $\E W^2$
  satisfying
  \begin{equation*}
    \E[W'-W|W] = -\lambda(W-R)
  \end{equation*}
  for some $\lambda \in (0,1)$ and some random variable $R$ and $|W'-W| \leq \delta$, then
  \begin{equation*}
    \begin{split}
      \sup_{z \in \mathbb{R}} |P(W \leq z) - \Phi(z)|
      &\leq \frac{.41 \delta^3}{\lambda} + 3 \delta (\sqrt{\E W^2} + \E |R|)
      + \frac{1}{2\lambda} \sqrt{\var (\E [(W'-W)^2|W])} \\
      &\quad + |\E W^2 - 1| + \E |WR| + \E |R|
    \end{split}
  \end{equation*}
\end{theorem}
