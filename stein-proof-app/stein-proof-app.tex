\chapter{Rate of Convergence Bounds}
\label{A:stein-proof-app}
\subsection{Proof of Proposition~\ref{P:P1}}
\begin{proof}
  \begin{align}
    \E T_{\Pi}^2
    &= \frac{N-1}{N} \E \left [ \left ( \frac{q_\Pi}{d_\Pi} \right )^2 \right ] \\
    &= \frac{N-1}{N} \E \left [\frac{
        4 N^2\bar{u}_{2,\Pi}^2}{2N-2N\bar{u}_{2,\Pi}^2} \right ] \quad \text{ from } \eqref{eq:qpi}
    \nonumber \\
    &= 2(N-1)\E \left [\frac{
        \bar{u}_{2,\Pi}^2}{1-\bar{u}_{2,\Pi}^2} \right ] \nonumber \\
    &= 2(N-1) \E g(\bar{u}_{2,\Pi}), \label{def:varT}
  \end{align}
  where $g(x) = \frac{x^2}{1-x^2}$. Now we proceed to calculate moments of $\bar{u}_{2,\Pi}$.

  Mean-centering the $u_{i}$ has the effect of mean-centering $\bar{u}_{2,\Pi}$:
  \begin{equation*}
    \E \bar{u}_{2,\Pi} =
    \frac{1}{N} \E \left [ \sum_{i=N+1}^{2N} u_{\Pi(i)} \right ] =
    \frac{1}{N} \sum_{i=N+1}^{2N} \E u_{\Pi(i)} =
    \frac{1}{N} \sum_{i=N+1}^{2N} \frac{1}{2N} \sum_{j=1}^{2N} u_{j} = 0
  \end{equation*}
  Under independence, $\var(\bar{u}_{2,\Pi})$ would be $\frac{1}{N}$ given
  the scaling.  However, the negative dependence induced by the
  permutation structure approximately halves this value.
  The scaling is such that $\var(u_{\Pi(i)}) = 1$.  Under independence and
  with $i \neq j$, $\var(u_{\Pi(i)} + u_{\Pi(j)}) = 2$.  Summing only 2 (out of $2N$)
  values under permutation dependence, $\var(u_{\Pi(i)} + u_{\Pi(j)}) = 2 - \frac{2}{2N-1}$.

  We can't use Serfling's result here because we need more than just an upper bound.
  \begin{align*}
    \var(\bar{u}_{2,\Pi})
    &= \frac{1}{N^2} \E \left [\left (\sum_{i=N+1}^{2N}u_{\Pi(i)}\right )^2\right ] \\
    &= \frac{1}{N^2} \E \left [\sum_{i=N+1}^{2N} u_{\Pi(i)}^2 +
    \sum_{i=N+1}^{2N}\sum_{j=N+1, j\neq i}^{2N}u_{\Pi(i)}u_{\Pi(j)} \right ] \\
    &= \frac{1}{N^2} \sum_{i=N+1}^{2N} \frac{1}{2N} \sum_{j=1}^{2N} u_{j}^2
    + \frac{1}{N^2}\sum_{i=N+1}^{2N}\sum_{j=N+1, j\neq
      i}^{2N} \E[u_{\Pi(i)} u_{\Pi(j)}] \\
    &= \frac{1}{N} + \frac{1}{N^2}\sum_{i=N+1}^{2N}\sum_{j=N+1, j\neq
      i}^{2N} \frac{1}{2N}\frac{1}{2N-1}\sum_{k=1}^{2N}
    \sum_{l=1, l\neq k}^{2N} u_ku_l \\
    &= \frac{1}{N} + \frac{1}{N^2}\sum_{i=N+1}^{2N}\sum_{j=N+1, j\neq
      i}^{2N} \frac{1}{2N}\frac{1}{2N-1} \left (
    \left (\sum_{k=1}^{2N} u_k \right )^2 - \sum_{k=1}^{2N} u_k^2
  \right ) \\
    &= \frac{1}{N} + \frac{1}{N^2}\sum_{i=N+1}^{2N}\sum_{j=N+1, j\neq
      i}^{2N} \frac{1}{2N}\frac{1}{2N-1} \left (
    0^2 - 2N \right ) \\
    &= \frac{1}{N} + \frac{1}{N}(N^2-N)\left (-\frac{1}{2N-1}\right ) \\
    &= \frac{2N-1}{N(2N-1)} + \frac{1-N}{N(2N-1)} \\
    &= \frac{1}{2N-1}
  \end{align*}

  Having established the first two moments, we compute the third degree Taylor expansion and bound
  the error in the approximation. By Taylor's theorem, we expand the function $g(\bar{u}_{2,\Pi}) =
  \frac{\bar{u}_{2,\Pi}^2}{1-\bar{u}_{2,\Pi}^2}$ around $\E[\bar{u}_{2,\Pi}] = 0$:
  \begin{equation*}
    g(\bar{u}_{2,\Pi}) = \frac{\bar{u}_{2,\Pi}^2}{1-\bar{u}_{2,\Pi}^2} = g(0) + g'(0)
    \bar{u}_{2,\Pi} + \frac{g''(0)}{2!}\bar{u}_{2,\Pi}^2 +
    \frac{g^{(3)}(0)}{3!}\bar{u}_{2,\Pi}^3 + R_3(\bar{u}_{2,\Pi}),
  \end{equation*}
  where $R_3(\bar{u}_{2,\Pi}) = \frac{g^{(4)}(\xi_L)}{4!}\bar{u}_{2,\Pi}^4$, with
  $\xi_L \in [0, \bar{u}_{2,\Pi}]$.

  From (\ref{def:varT}) and evaluating the Taylor series, we have
  \begin{equation*}
    \E g(\bar{u}_{2,\Pi}) = \frac{\E T_{\Pi}^2}{2(N-1)} = \E[\bar{u}_{2,\Pi}^2 + R_3(\bar{u}_{2,\Pi})].
  \end{equation*}
  Therefore,
  \begin{align*}
    \left | \frac{\E T_{\Pi}^2}{2(N-1)} - \E \bar{u}_{2,\Pi}^2 \right |
    &= \left | \frac{\E T_{\Pi}^2}{2(N-1)} - \frac{1}{2N-1} \right | \\
    &\leq  \E |R_3(\bar{u}_{2,\Pi})| \\
    &= \E \left | \frac{24(5\xi_L^4 + 10 \xi_L^2 +
        1)}{4!(\xi_L-1)^5}\bar{u}_{2,\Pi}^4 \right | \\
    &\leq \E \left | \frac{24(5\bar{u}_{2,\Pi}^4 + 10 \bar{u}_{2,\Pi}^2 +
        1)}{4!(\bar{u}_{2,\Pi}-1)^5}\bar{u}_{2,\Pi}^4 \right | \\
    &\leq \frac{5B^4+10B^2+1}{|B-1|^5} \E \bar{u}_{2,\Pi}^4 \\
    &\leq \frac{5B^4+10B^2+1}{|B-1|^5} f_{c_1}(4) N^{-2} \quad \text{
      by } (\ref{def:serfling}) \\
    &:= c_1 N^{-2}
  \end{align*}

  \begin{align*}
    |\E T_{\Pi}^2 - 1| - \frac{1}{2N-1}
    &\leq \left | \E T_{\Pi}^2 - 1 + \frac{1}{2N-1} \right | \\
    &= \left | \E T_{\Pi}^2 - \frac{2(N-1)}{2N-1} \right | \\
    &= 2(N-1) \left | \frac{\E T_{\Pi}^2}{2(N-1)} - \frac{1}{2N-1} \right | \\
    &\leq c_1 2(N-1)N^{-2}
  \end{align*}

  This implies that
  \begin{equation*}
    |\E T_{\Pi}^2 - 1| \leq \frac{1}{2N-1} + c_1 \frac{2N-2}{N^2} \leq
    \frac{1 + 2c_1}{N} := c_2 N^{-1}
  \end{equation*}
\end{proof}

\subsection{Proof of Proposition~\ref{P:P2}}
\begin{proof}
  With two applications of the $c_r$ inequality, we can bound the variance of the sum by a constant
times the sum of the variances.  Suppose $X$, $Y$, and $Z$ have finite variances.  Then, with the
centered random variables represented by $\tilde{X}, \tilde{Y}$, and $\tilde{Z}$, we have that
  \begin{align*}
    \var(X + Y + Z)
    &= \var(\tilde{X}+\tilde{Y}+\tilde{Z}) \\
    &= \E|(\tilde{X}+\tilde{Y})+\tilde{Z}|^2 \\
    &\leq 2 \E |\tilde{X}+\tilde{Y}|^2 + 2\E |\tilde{Z}|^2 \\
    &\leq 2(2\E \tilde{X}^2 + 2\E \tilde{Y}^2) + 2\E \tilde{Z}^2 \\
    &\leq 4(\var(X) + \var(Y) + \var(Z))
  \end{align*}

  From \eqref{def:ttpcubed},
  \begin{align*}
    \var (\E[(T_{\Pi}'-T_{\Pi})^2 | \Pi = \pi]) &= \var \left ( \frac{N-1}{N}\E \left [
      \left ( \frac{2u_{\Pi(J)}-2u_{\Pi(I)}}{d_{\Pi}}+T_{\Pi}'\frac{d_{\Pi}-d_{\Pi}'}{d_{\Pi}} \right )^2
        \middle | \Pi = \pi \right ] \right ) \\
    &\leq \var \left ( \E \left [
      \left ( \frac{2u_{\Pi(J)}-2u_{\Pi(I)}}{d_{\Pi}}+T_{\Pi}'\frac{d_{\Pi}-d_{\Pi}'}{d_{\Pi}} \right )^2
        \middle | \Pi = \pi \right ] \right ) \\
    &\leq 4(\var(X) + \var(Y) + \var(Z))
  \end{align*}
  where
  \begin{align*}
    X &= \E \left [ \left ( \frac{2u_{\Pi(J)}-2u_{\Pi(I)}}{d_{\Pi}} \right )^2
        \middle | \Pi = \pi \right ] \\
    Y &= \E \left [ \left
          ( T_{\Pi}'\frac{d_{\Pi}-d_{\Pi}'}{d_{\Pi}} \right )^2 \middle | \Pi = \pi \right ] \\
    Z &= 2 \E \left [ \left ( \frac{2u_{\Pi(J)}-2u_{\Pi(I)}}{d_{\Pi}}
         T_{\Pi}'\frac{d_{\Pi}-d_{\Pi}'}{d_{\Pi}} \right ) \middle | \Pi = \pi \right ]
  \end{align*}
  The $X$ term will dominate, so we can afford to use coarser methods on $Y$ and $Z$.

  The $\E[u_{\Pi(J)}-u_{\Pi(I)} | \Pi = \pi]$ term is common to applications of Stein's method of
  exchangeable pairs.  However, there is a complication in the $d_{\Pi}$ random variable in the
  denominator.  Our strategy will be to calculate the two variances separately with some necessary
  additional terms.

  First, we prove an intermediate result regarding the variance of a product of random variables
  \begin{equation*}
    W = (d_{\Pi})^{-2} \text{ and } V = \E [(u_{\Pi(J)} - u_{\Pi(I)})^2 | \Pi = \pi].
  \end{equation*}
  Then $\var(X) = 4 \var(WV)$ since $d_{\Pi}$ is $\sigma(\Pi)$-measurable and
  \begin{align}
    \label{E:var_prod}
    \var(WV) &= \var(W(V-\E V) + W \E V) \nonumber \\
    &\leq 2\var(W(V-\E V)) + 2 \var (W \E V) \nonumber \\
    &\leq 2 \E [W^2 (V-\E V)^2] + 2(\E V)^2 \var(W) \nonumber \\
    &\leq 2(f_{c_2}(2))^2 N^{-2} \var(V) + 2u_{\Delta}^4 \var(W).
  \end{align}

  \begin{align*}
    \var(W) &= \var ((d_{\Pi})^{-2}) \\
    &= \var \left ( \frac{1}{2N(1-\bar{u}_{2,\Pi}^2)} \right ) \\
    &= \frac{1}{4N^2} \left [ \E \left [ \left (\frac{1}{1-\bar{u}_{2,\Pi}^2}
        \right )^2 \right ] -
      \left ( \E \left [ \frac{1}{1-\bar{u}_{2,\Pi}^2} \right ] \right )^2
    \right ] \\
    &= \frac{1}{4N^2}[\E h(\bar{u}_{2,\Pi}) - (\E \tilde{h}(\bar{u}_{2,\Pi}))^2],
  \end{align*}
  where
  \begin{equation*}
    h(x) = \left ( \frac{1}{1-x^2} \right )^2 = 1 + 2x^2 + 3x^4 + \dots
      \text{ and } \tilde{h}(x) = \frac{1}{1-x^2} = 1 + x^2 + x^4 + \ldots
  \end{equation*}

  By Taylor's theorem,
  \begin{equation*}
    \E \left [ \left (\frac{1}{1-\bar{u}_{2,\Pi}^2} \right )^2 \right ]
    = 1 + 2 \left ( \frac{1}{2N-1} \right ) + \E R_3(\bar{u}_{2,\Pi}),
  \end{equation*}
  with
  \begin{equation*}
    \E |R_3(\bar{u}_{2,\Pi})| \leq \frac{24(35B^4 + 42 B^2 + 3)}{4!(B-1)^6}f_{c_1}(4) N^{-2}
    := c_4 N^{-2}
  \end{equation*}
  Re-arranging, we get
  \begin{equation*}
    \left | \E \left [ \left (\frac{1}{1-\bar{u}_{2,\Pi}^2} \right )^2 \right ]
      - 1 - \frac{2}{2N-1} \right | \leq c_4 N^{-2}.
  \end{equation*}

  Applying Taylor's theorem to $\tilde{h}$:
  \begin{equation*}
    \E \left [ \frac{1}{1-\bar{u}_{2,\Pi}^2} \right ]
    = 1 + \frac{1}{2N-1} + \E \tilde{R}_3(\bar{u}_{2,\Pi}),
  \end{equation*}
  with
  \begin{equation*}
    \E |\tilde{R}_3(\bar{u}_{2,\Pi})| \leq \frac{24(5B^4 + 10B^2 + 1)}{4!(B-1)^5}f_{c_1}(4) N^{-2}
    := c_5 N^{-2}
  \end{equation*}
  Squaring, applying the bound, and re-arranging yields
  \begin{equation*}
    \left | \left ( \E \left [ \frac{1}{1-\bar{u}_{2,\Pi}^2} \right ] \right )^2
      - \left ( 1 + \frac{1}{2N-1} \right )^2 \right | \leq
    2 \left ( 1 + \frac{1}{2N-1} \right ) c_5 N^{-2} + c_5^2 N^{-4}
  \end{equation*}

  Now we combine bounds to get
  \begin{align*}
    &\left | \E \left [ \left (\frac{1}{1-\bar{u}_{2,\Pi}^2} \right )^2 \right ] -
      \left ( \E \left [ \frac{1}{1-\bar{u}_{2,\Pi}^2} \right ] \right )^2 \right | \\
    &= \left | \E \left [ \left (\frac{1}{1-\bar{u}_{2,\Pi}^2} \right )^2 \right ] -
      \left ( \E \left [ \frac{1}{1-\bar{u}_{2,\Pi}^2} \right ] \right )^2
    + \frac{1}{(2N-1)^2} - \frac{1}{(2N-1)^2} \right | \\
  &\leq \left | \E \left [ \left (\frac{1}{1-\bar{u}_{2,\Pi}^2} \right )^2 \right ] -
      \left ( \E \left [ \frac{1}{1-\bar{u}_{2,\Pi}^2} \right ] \right )^2
      + \frac{1}{(2N-1)^2} \right | + \left | \frac{1}{(2N-1)^2} \right |\\
    &\leq \left |
      \E \left [ \left (\frac{1}{1-\bar{u}_{2,\Pi}^2} \right )^2 \right ]
      - 1 - \frac{2}{2N-1} -
      \left (
        \left ( \E \left [ \frac{1}{1-\bar{u}_{2,\Pi}^2} \right ] \right )^2
        - \left ( 1 + \frac{1}{2N-1} \right )^2
      \right ) \right | + \left | \frac{1}{(2N-1)^2} \right | \\
    &\leq c_4 N^{-2}
    + 2 \left ( 1 + \frac{1}{2N-1} \right ) c_5 N^{-2} + c_5^2 N^{-4}
    + \left | \frac{1}{(2N-1)^2} \right | \\
    &\leq (c_4 + 3c_5 + c_5^2 + \frac{1}{4}) N^{-2} \\
    &:= c_6 N^{-2}
  \end{align*}
  Therefore, $\var(W) \leq \frac{c_6}{4}N^{-4}$ and
  \begin{equation*}
    \var(X) \leq 8(f_{c_2}(2))^2 N^{-2} \var(V) + 8u_{\Delta}^4 \frac{c_6}{4} N^{-4}
  \end{equation*}
  with
  \begin{align*}
    \var(V) &= \var (\E[(u_{\Pi(J)}-u_{\Pi(I)})^2 | \Pi = \pi]) \\
    &= \var(\E[u_{\Pi(J)}^2+u_{\Pi(I)}^2-2u_{\Pi(J)}u_{\Pi(I)}| \Pi = \pi]) \\
    &= \var \left ( \frac{1}{N^2} \sum_{I=1}^{N}\sum_{J=N+1}^{2N}
    (u_{\pi(J)}^2+u_{\pi(I)}^2-2u_{\pi(J)}u_{\pi(I)}) \right ) \\
    &= \var \left ( \frac{1}{N^2} \left ( N \sum_{K=1}^{2N}u_K^2 -
      \sum_{I=1}^{N}\sum_{J=N+1}^{2N}2u_{\pi(J)}u_{\pi(I)} \right ) \right ) \\
    &= \frac{4}{N^4}
    \sum_{I=1}^{N}\sum_{J=N+1}^{2N}\sum_{K=1}^{N}\sum_{L=N+1}^{2N}
    \cov(u_{\pi(I)}u_{\pi(J)},u_{\pi(K)}u_{\pi(L)})
  \end{align*}
  since $\sum_{K=1}^{2N}u_K^2 = 2N$ is a constant.  We proceed by calculating
  \begin{equation*}
    \cov(u_{\pi(I)}u_{\pi(J)},u_{\pi(K)}u_{\pi(L)})
    = \E[u_{\pi(I)}u_{\pi(J)}u_{\pi(K)}u_{\pi(L)}] - \E[u_{\pi(I)}u_{\pi(J)}]\E[u_{\pi(K)}u_{\pi(L)}].
  \end{equation*}
  The index sets for variables $I$ and $J$ (and $K$ and $L$) are disjoint, so
  \begin{equation*}
    \E[u_{\pi(I)}u_{\pi(J)}] = \E[u_{\pi(K)}u_{\pi(L)}]
    = \frac{1}{2N}\frac{1}{2N-1}\sum_{I=1}^{2N} u_{I} \sum_{J=1,J\neq I}^{2N} u_{J} = -\frac{1}{2N-1}
  \end{equation*}
  for all values of $I, J, K, L$ in the sum.  Therefore,
  \begin{equation*}
    \E[u_{\pi(I)}u_{\pi(J)}] = \E[u_{\pi(K)}u_{\pi(L)}] = \frac{1}{(2N-1)^2}.
  \end{equation*}
  However, $K$ could equal $I$ and $L$ could equal $J$, which changes the mass assigned by the
  permutation distribution, necessitating a separate treatment for each case.

  Case $I \neq J \neq K \neq L$:
  \begin{align*}
    &\E[u_{\pi(I)} u_{\pi(J)} u_{\pi(K)} u_{\pi(L)}] \\
    &= \frac{1}{2N}\frac{1}{2N-1}\frac{1}{2N-2}\frac{1}{2N-3}
    \sum_{I=1}^{2N} \sum_{J=1, J\neq I}^{2N} \sum_{K=1, K \neq I,J}^{2N} \sum_{L=1, L \neq I,J,K}^{2N}
    u_{I} u_{J} u_{K} u_{L} \\
    &= \frac{1}{2N}\frac{1}{2N-1}\frac{1}{2N-2}\frac{1}{2N-3}
    \sum_{I=1}^{2N} u_{I} \sum_{J=1, J\neq I}^{2N} u_{J} \sum_{K=1, K \neq
      I,J}^{2N} u_{K} (-u_{I}-u_{J}-u_{K}) \\
    &= \frac{1}{2N}\frac{1}{2N-1}\frac{1}{2N-2}\frac{1}{2N-3}
    \sum_{I=1}^{2N} u_{I} \sum_{J=1, J\neq I}^{2N} u_{J}
    ((-u_{I}-u_{J})(-u_{I}-u_{J})+(u_{I}^2+u_{J}^2-2N)) \\
    &= \frac{1}{2N}\frac{1}{2N-1}\frac{1}{2N-2}\frac{1}{2N-3}
    \sum_{I=1}^{2N} u_{I} \sum_{J=1, J\neq I}^{2N} u_{J}
    (2u_{I}^2-2N+2u_{J}^2+2u_{I}u_{J}) \\
    &= \frac{1}{2N}\frac{1}{2N-1}\frac{1}{2N-2}\frac{1}{2N-3}
    \sum_{I=1}^{2N} u_{I} \left (
      (2u_{I}^2-2N)(-u_{I})+2\sum_{J=1, J\neq I}^{2N} u_{J}^3+2u_{I}(2N-u_{I}^2)
    \right) \\
    &= \frac{1}{2N}\frac{1}{2N-1}\frac{1}{2N-2}\frac{1}{2N-3}
    \sum_{I=1}^{2N} u_{I} \left (
      -4u_{I}^3 + 6Nu_{I} + 2\left (\sum_{J=1}^{2N}u_{J}^3 - u_{I}^3 \right )
    \right) \\
    &= \frac{1}{2N}\frac{1}{2N-1}\frac{1}{2N-2}\frac{1}{2N-3}
    \left ( -6\sum_{I=1}^{2N}u_{I}^4+12N^2 \right )
  \end{align*}
  for $N^2(N-1)^2$ terms in the sum.

  Case $I=K$ and $J=L$:
  \begin{align*}
    \E[u_{\pi(I)}^2 u_{\pi(J)}^2] &= \frac{1}{2N}\frac{1}{2N-1} \sum_{I=1}^{2N}
    \sum_{J=1, J\neq I}^{2N} u_{I}^2 u_{J}^2 \\
    &= \frac{1}{2N}\frac{1}{2N-1} \sum_{I=1}^{2N} u_{I}^2 (2N - u_{I}^2) \\
    &= \frac{2N}{2N-1}-\frac{1}{2N}\frac{1}{2N-1} \sum_{I=1}^{2N} u_{I}^4
  \end{align*}
  for $N^2$ terms in the sum.

  Case $I=K, J\neq L$ or $I\neq K, J= L$:
  \begin{align*}
    \E[u_{\pi(I)}^2 u_{\pi(J)} u_{\pi(K)}] &= \frac{1}{2N}\frac{1}{2N-1}\frac{1}{2N-2} \sum_{I=1}^{2N}
    \sum_{J=1, J\neq I}^{2N} \sum_{K=1, K \neq I,J}^{2N} u_{I}^2 u_{J} u_{K} \\
    &=\frac{1}{2N}\frac{1}{2N-1}\frac{1}{2N-2} \sum_{I=1}^{2N}
    \sum_{J=1, J\neq I}^{2N} u_{I}^2 u_{J} (0 - u_{I} - u_{J}) \\
    &=-\frac{1}{2N}\frac{1}{2N-1}\frac{1}{2N-2} \left(
      \sum_{I=1}^{2N} u_{I}^3 \sum_{J=1, J\neq I}^{2N} u_{J} +
      \sum_{I=1}^{2N} u_{I}^2 \sum_{J=1, J\neq I}^{2N} u_{J}^2
    \right ) \\
    &=-\frac{1}{2N}\frac{1}{2N-1}\frac{1}{2N-2} \left(
      \sum_{I=1}^{2N} - u_{I}^4 +
      \sum_{I=1}^{2N} u_{I}^2 (2N - u_{I}^2)
    \right ) \\
    &=\frac{1}{2N}\frac{1}{2N-1}\frac{1}{2N-2} \left(
      2 \sum_{I=1}^{2N} u_{I}^4 -4N^2 \right )
  \end{align*}
  for $2N^2(N-1)$ terms in the sum.

  Putting it all together, we have
  \begin{align*}
    &\var (\E[(u_{\Pi(J)}-u_{\Pi(i)})^2] | \Pi = \pi) \\
    &= \frac{4}{N^4} (N^2(N-1)^2)
    \left ( \frac{1}{(2N)(2N-1)(2N-2)(2N-3)}
      \left ( -6\sum_{i=1}^{2N}u_{i}^4 + 12N^2 \right ) - \frac{1}{(2N-1)^2} \right ) \\
    &+ \frac{4}{N^4} N^2 \left ( \frac{2N}{2N-1} -
      \frac{1}{2N}\frac{1}{2N-1}\sum_{i=1}^{2N}u_{i}^4 - \frac{1}{(2N-1)^2}
    \right ) \\
    &+ \frac{4}{N^4} (2N^2(N-1)) \left (
      \frac{1}{2N}\frac{1}{2N-1}\frac{1}{2N-2}\left (
        2 \sum_{i=1}^{2N} u_{i}^4 - 4N^2 \right ) - \frac{1}{(2N-1)^2}
    \right ) \\
    &\leq \frac{48}{4N^2} + \frac{8}{N^2} + \frac{16 \sum_{i=1}^{2N}
      u_{i}^4}{N^4} \\
    &= \left ( 20 + 16 \left ( \sum_{i=1}^{2N} u_{i}^4 \right )N^{-2} \right ) N^{-2}
  \end{align*}

  Therefore,
  \begin{equation*}
    \var(X) \leq 8(f_{c_2}(2))^2 \left ( 20 + 16 \left ( \sum_{i=1}^{2N} u_{i}^4 \right )N^{-2}
    \right ) N^{-4}
    + 8u_{\Delta}^4 \frac{c_6}{4} N^{-4}
  \end{equation*}

  Because the latter two terms are much smaller in order, we can apply
  coarser techniques.  In particular, we use the following bound:
  \begin{equation*}
    \var (\E[U|V]) = \var(U) - \E (\var (U|V)) \leq E[U^2]
  \end{equation*}

  Applying to the second term,
  \begin{align*}
    \var(Y) &= \var \left ( \E \left [ \left ( T'_{\Pi}\frac{d_{\Pi}-d_{\Pi}'}{d_{\Pi}} \right )^2
        \middle | \Pi = \pi \right ] \right ) \\
    &\leq \E \left [ \left ( \frac{q_{\Pi}'}{d_{\Pi}d_{\Pi}'} (d_{\Pi}-d_{\Pi}') \right )^4 \right ] \\
    &\leq \sqrt{\E \left [ \left ( \frac{q_{\Pi}'}{d_{\Pi}d_{\Pi}'} \right )^8 \right ]
      \E [(d_{\Pi}-d_{\Pi}')^8]} \\
    &\leq \sqrt{f_{c_6}(8) N^{-8/2} f_{c_4}(8) N^{-8}} \text{ from }
    (\ref{def:qpddp}),(\ref{def:ddiffp}) \\
    &= \sqrt{f_{c_6}(8)f_{c_4}(8)}N^{-6} \\
    &:= c_7 N^{-6}
  \end{align*}

  And to the third,
  \begin{align*}
    \var(Z) &= 4 \var \left ( \E \left [ \left ( \frac{2u_{\Pi(J)}-2u_{\Pi(I)}}{d_{\Pi}}
          T_{\Pi}'\frac{d_{\Pi}-d_{\Pi}'}{d_{\Pi}} \right ) \middle | \Pi = \pi \right ] \right ) \\
    &\leq 16 u_{\Delta}^2 \E \left [ \left ( \frac{1}{d_{\Pi}} \frac{q_{\Pi}'}{d_{\Pi}d_{\Pi}'}
        (d_{\Pi}-d_{\Pi}') \right )^2 \right ] \\
    &\leq 16 u_{\Delta}^2 f_{c_2}(2) N^{-2/2} \sqrt{\E \left [ \left (
          \frac{q_{\Pi}'}{d_{\Pi}d_{\Pi}'} \right )^4 \right ] \E[(d_{\Pi}-d_{\Pi}')^4]
    } \text{ from } (\ref{def:dp}) \\
    &\leq 16 u_{\Delta}^2 f_{c_2}(2) N^{-1} \sqrt{f_{c_6}(4) N^{-4/2} f_{c_4}(4) N^{-4}}
    \text{ from } (\ref{def:qpddp}),(\ref{def:ddiffp}) \\
    &\leq 16 u_{\Delta}^2 f_{c_2}(2) (f_{c_6}(4))^{-1/2} (f_{c_4}(4))^{-1/2} N^{-4} \\
    &:= c_8 N^{-4}
  \end{align*}

  \begin{align*}
    &\frac{1}{2\lambda} \sqrt{\var (\E [(T_{\Pi}'-T_{\Pi})^2|T_{\Pi}])} \\
    &= N \sqrt{(\var(X) + \var(Y) + \var(Z))} \\
    &\leq N \sqrt{
      8(f_{c_2}(2))^2 \left ( 20 + 16 \left ( \sum_{i=1}^{2N} u_{i}^4 \right )N^{-2} \right ) N^{-4}
      + 8u_{\Delta}^4 \frac{c_6}{4} N^{-4}
      + c_7 N^{-6} + c_8 N^{-4}} \\
    &:= N^{-1} c_3\sqrt{20 + 16\frac{\sum_{i=1}^{2N} u_{i}^4}{N^2}} \\
  \end{align*}
\end{proof}

\subsection{Proof of Proposition~\ref{P:P3}}
\begin{proof}
  The strategy is to break apart the remainder term from the main piece.  From (\ref{def:ttpcubed}),
  \begin{align*}
    \E|T_{\Pi}'-T_{\Pi}|^3
    &= \left (\frac{N-1}{N}\right )^{3/2}
    \E \left [d_{\Pi}^{-3} \left |2u_{\Pi(J)}-2u_{\Pi(I)}+q_{\Pi}'\frac{d_{\Pi}-d_{\Pi}'}{d_{\Pi}'} \right |^3
    \right ] \\
    &\leq 8 \left (
      8 u_{\Delta}^3 \E [d_{\Pi}^{-3}] +
      \sqrt{\E \left [ \left ( \frac{q_{\Pi}'}{d_{\Pi}d_{\Pi}'} \right )^6 \right ]  \E
        [(d_{\Pi}-d_{\Pi}')^6]} \right ) \\
    &\leq 64 u_{\Delta}^3 f_{c_2}(3) N^{-3/2} +
    8 \sqrt{f_{c_6}(6) N^{-6/2} f_{c_4}(6) N^{-6}} \text{ from }
    (\ref{def:dp}),(\ref{def:qpddp}),(\ref{def:ddiffp}) \\
    &\leq \frac{c_9^2}{2} N^{-3/2}
  \end{align*}
  Therefore,
  \begin{equation*}
    (2\pi)^{-1/4}\sqrt{\frac{\E|T_{\Pi}'-T_{\Pi}|^3}{\lambda}} \leq
    (2\pi)^{-1/4} c_9 N^{-1/4}.
  \end{equation*}
\end{proof}

\subsection{Proof of Proposition~\ref{P:P4}}
\begin{proof}
  \begin{align*}
    \E|R| &= \E \left | \left (\frac{N}{2}\right )
      \sqrt{\frac{N-1}{N}}\frac{1}{d_{\Pi}}\E
      \left [q_{\Pi}'\frac{(d_{\Pi}-d_{\Pi}')}{d_{\Pi}'} \middle | T_{\Pi} \right ]\right | \\
    &\leq \frac{N}{2} \E \left | \frac{q_{\Pi}'}{d_{\Pi}d_{\Pi}'}(d_{\Pi}-d_{\Pi}') \right | \\
    &\leq \frac{N}{2} \sqrt{\E \left | \frac{q_{\Pi}'}{d_{\Pi}d_{\Pi}'} \right |^2
      \E[d_{\Pi}-d_{\Pi}']^2} \\
    &\leq \frac{N}{2} \sqrt{f_{c_6}(2) N^{-2/2} f_{c_4}(2) N^{-2}}
    \text{ from } (\ref{def:qpddp}), (\ref{def:ddiffp}) \\
    &=\frac{1}{2}\sqrt{f_{c_6}(2)f_{c_4}(2)} N^{-1/2}
  \end{align*}
\end{proof}

\subsection{Proof of Proposition~\ref{P:P5}}
\begin{proof}
  \begin{align*}
    \E|T_{\Pi}R| &= \E \left | T_{\Pi} \left (\frac{N}{2}\right )
      \sqrt{\frac{N-1}{N}}\frac{1}{d_{\Pi}}\E
      \left [q_{\Pi}'\frac{(d_{\Pi}-d_{\Pi}')}{d_{\Pi}'} \middle | T_{\Pi} \right ]\right | \\
    &\leq \frac{N}{2} \E \left |T_{\Pi} \frac{q_{\Pi}'}{d_{\Pi}d_{\Pi}'}(d_{\Pi}-d_{\Pi}') \right | \\
    &\leq \frac{N}{2} \sqrt{\E T_{\Pi}^2 \E \left [ \left ( \frac{q_{\Pi}'}{d_{\Pi}d_{\Pi}'} \right )^2
        (d_{\Pi}-d_{\Pi}')^2 \right ] } \\
    &\leq \frac{N}{2} \sqrt{\E T_{\Pi}^2 \sqrt{\E \left [ \left ( \frac{q_{\Pi}'}{d_{\Pi}d_{\Pi}'}
          \right )^4 \right ] \E[(d_{\Pi}-d_{\Pi}')^4] }} \\
    &\leq \frac{N}{2} \sqrt{\E T_{\Pi}^2 \sqrt{f_{c_6}(4) N^{-4/2} f_{c_4}(4) N^{-4}}}
    \text{ from } (\ref{def:qpddp}), (\ref{def:ddiffp}) \\
    &=\frac{N^{-1/2}}{2}(f_{c_6}(4)f_{c_4}(4))^{1/4}\sqrt{\E T_{\Pi}^2} \\
    &\leq \frac{1}{2}(f_{c_6}(4)f_{c_4}(4))^{1/4} \sqrt{2+2c_1} N^{-1/2}
  \end{align*}
  because $\E T_{\Pi}^2 \leq 1 + \frac{1+2c_1}{N} \leq 2 + 2c_1$.
\end{proof}