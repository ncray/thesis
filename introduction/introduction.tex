\chapter{Introduction}

The two-sample testing problem arises whenever one collects data for
two distinct groups and wishes to know whether there are meaningful
differences between such groups: could they have been sampled from the
same underlying distribution?  Throughout statistical history, many
two-sample tests have been developed to address a multitude of
scenarios.  However, there is a dearth of scholarship in the
heterogeneous data setting, where a single sample is comprised of data
from different domains.

We develop Jerome Friedman's method of conducting two-sample tests
by leveraging techniques from regression and classification.  In
particular, we construct two-sample tests for non-vectorial data by
way of kernel support vector machines.  We extend these tests to deal
with heterogeneous data by using multiple kernel learning to create
an optimal single kernel as a convex combination of various input
kernels.  We show that these tests generalize the two-sample permutation
$t$-test, for which asymptotic normality results are known but rate of
convergence bounds remain lacking.  Using Stein's method of
exchangeable pairs, we produce Berry--Esseen-type bounds that allow us
to quantify our error in approximating the permutation distribution
of this statistic with the normal distribution.

Chapter~\ref{C:steins-method} gives an overview of Stein's method,
with an emphasis on exchangeable pairs.  We review a similar CLT with
dependence, Hoeffding's Combinatorial CLT, as our proof proceeds in a
similar fashion.  Ancillary results can be found in
Appendix~\ref{A:auxiliary-app}, generalizations of previous work on
Stein's method in Appendix~\ref{A:steins-method-app}, and proofs of
the main propositions contributing to our rate of convergence bounds
are in Appendix~\ref{A:stein-proof-app}

In Chapter~\ref{C:stein-proof}, we review population- and
permutation-based inference and prior related work on distributional
approximations for the $t$-statistic, as well as its connection with
self-normalized sums.  We produce a conservative bound on rate of
convergence of the permutation $t$-distribution to the standard normal
distribution, in addition to a Berry--Esseen-type bound given an
additional boundedness condition, appealing to some results that are
proved in Appendix~\ref{A:stein-proof-app}.

Chapter~\ref{C:simulations} is a computational companion to
Chapter~\ref{C:stein-proof}.  We collate the results from numerous
computer simulations to experimentally verify the theoretical results
and inform the development of further theory.  We also describe some
novel computational procedures in order to effect simulation over a
wide range of conditions.

We introduce Friedman's idea for two-sample testing in
Chapter~\ref{C:friedman-test} and apply it using kernel support vector
machines, developing two-sample tests for situations for which
kernel functions exist.  We show that these tests generalize the
permutation $t$-test and compare their performance against an
alternative, kernel-based test utilizing the maximum mean discrepancy
statistic in vectorial and non-vectorial settings.

Lastly, in Chapter~\ref{C:MKL}, we construct two-sample tests on
heterogeneous data using multiple kernel learning to identify the most
salient discriminating features of the data.  We demonstrate their
efficacy and informativeness on wine data and are able to discriminate
between two popular wine varietals by reliably fusing data from
different domains.