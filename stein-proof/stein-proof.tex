\chapter{Main Proof}
\label{C:stein-proof}
In this chapter, we prove the core theoretical result of this thesis, a rate of convergence bound for
the randomization distribution, using the theorem of chapter~\ref{C:steins-method}.

\section{Set-up}
We observe two samples with equal sample size: $\{u_i\}_{i=1}^N$ and $\{u_i\}_{i=N+1}^{2N}$.  

Student's two-sample $t$-statistic is given by
\begin{align*}
T(\{u_i\}_{i=1}^N, \{u_i\}_{i=N+1}^{2N}) 
&= \frac{\bar{u}_1 - \bar{u}_2}{\sqrt{\frac{\frac{1}{N-1}
      \sum_{i=1}^N(u_i - \bar{u}_1)^2}{N} + \frac{\frac{1}{N-1}
      \sum_{i=N+1}^{2N}(u_i - \bar{u}_2)^2}{N}}} \\
&= \frac{1}{\sqrt{\frac{N}{N-1}}} \frac{\sum_{i=1}^N u_i -
  \sum_{i=N+1}^{2N}u_i}{\sqrt{\sum_{i=1}^N(u_i - 
    \bar{u}_1)^2 + \sum_{i=N+1}^{2N}(u_i - \bar{u}_2)^2}} \\
&= \sqrt{\frac{N-1}{N}}\frac{q}{d},
\end{align*}
where
\begin{align*}
  q &= \left (\sum_{i=1, i\neq I}^N u_i + u_I -
    \sum_{i=N+1, i\neq J}^{2N}u_i - u_J\right ) \\
  d &= \sqrt{\sum_{i=1}^N(u_i - \bar{u}_1)^2 +
    \sum_{i=N+1}^{2N}(u_i - \bar{u}_2)^2}.
\end{align*}
Similarly, 
\begin{align*}
  T'(\{u_i\}_{i=1}^N, \{u_i\}_{i=N+1}^{2N}) 
  &= \sqrt{\frac{N-1}{N}}\frac{q'}{d'} \\
  q' &= \left (\sum_{i=1, i\neq I}^N u_i + u_J -
    \sum_{i=N+1, i\neq J}^{2N}u_i - u_I \right ) \\
  &= q - 2u_I + 2u_J \\
  d' &= \sqrt{\sum_{i=1}^N(u_i - \bar{u}'_1)^{2} +
    \sum_{i=N+1}^{2N}(u_i - \bar{u}'_2)^{2}}.  
\end{align*}

In order to perform hypothesis testing, we would like to know the randomization distribution of $T$.
We shall create an exchangeable pair $(T, T')$ by considering a uniformly random transposition $(I,
J)$.  WLOG, take $I \leq J$.  We apply this transposition to the group labels.  Note that if $I, J
\in \{1,\ldots,N\}$ or $I, J \in \{N+1,\ldots,2N\}$ then $T' = T$, where $T'$ is the $t$-statistic
under this random transposition.  That is, the $t$-statistic is invariant to within-group
transpositions.  Thus, the only changes occur when $1 \leq I \leq N$ and $N + 1 \leq J \leq 2N$.
With this in mind, let's redefine our transposition to be uniformly at random over the $N^2$ cases
where $1 \leq I \leq N$ and $N + 1 \leq J \leq 2N$.

\section{Assumptions}
Recall that the $t$-statistic is invariant up to sign under linear transformations, so we can
mean-center and scale so that $\sum_{i=1}^{2N} u_i = 0$ and $\sum_{i=1}^{2N} u_i^2 = 2N$.  The
proper transformation is 
\begin{equation}
  z_i = \sqrt{\frac{2N}{\sum (u_i - \bar{u})^2}}(u_i-\bar{u}), 
\end{equation}
so we just consider the $u_i$'s as having been transformed.  This can be seen as a very mild
assumption of disallowing the case where all our data are constant.  

We also assume that
\begin{equation}
  B = \max_{\pi} \bar{u}_2^2 < 1.
\end{equation}
This rejects situations like $(1, 1, \ldots, -1, -1)$.

\section{Preliminaries}
Here we collect useful bounds and other results.

In order to bound various moments of $\bar{u}_2$ under the permutation
distribution, we use a result of Serfling's
\cite{serfling1974probability}:
\begin{theorem}
  Consider sampling without replacement from a finite list of values
  $u_1, \ldots, u_{2N}$.  Let $a = \min_i u_i$ and $b = \max_i u_i$.
  Then for $p > 0$,
  \begin{align}
    \E [\bar{u}_2^p] 
    &\leq \frac{\Gamma(p/2 + 1)}{2^{p/2 + 1}}
    \left [ \frac{N+1}{2N}(b-a)^2 \right ]^{p/2}
    (2N)^{-p/2} \nonumber \\
    &\leq \frac{\Gamma(p/2 + 1)}{2^{p/2 + 1}}
    \left [ \frac{N+1}{4N}(b-a)^2 \right ]^{p/2}
    (N)^{-p/2} \nonumber \\
    &\leq \frac{\Gamma(p/2 + 1)}{2^{p/2 + 1}}
    \left [ \frac{1}{2}(b-a)^2 \right ]^{p/2}
    N^{-p/2} \nonumber \\
    &:= f_{c_1}(p)N^{-p/2} \label{def:serfling}.
  \end{align}
\end{theorem}

By assumption,
\begin{align}
  d^{-p} &= \frac{1}{(2N(1-\bar{u}_2^2))^{p/2}} \nonumber \\
  &\leq \frac{1}{(2N(1-B^2))^{p/2}} \nonumber \\
  &= \frac{1}{(2(1-B^2))^{p/2}}N^{-p/2} \nonumber \\
  &:= f_{c_2}(p) N^{-p/2} \label{def:dp}.
\end{align}

The transposition $(I, J)$ also affects the denominator of $T'$, and we need to quantify the
difference between the denominators of $T$ and $T'$.  
\begin{align*}
    d^2 &= \sum_{i=1}^N (u_i - \bar{u}_1)^2 + \sum_{i=N+1}^{2N} (u_i -
    \bar{u}_2)^2 = \sum_{i=1}^{2N} u_i^2 - N \bar{u}_1^2 - N
    \bar{u}_2^2 \\
    d'^2 &= \sum_{i=1}^{2N} u_i^2 - N \bar{u}_1'^2 - N \bar{u}_2'^2,  
\end{align*}
where
\begin{equation*}
  \label{eq:6}
  \bar{u}_1' = \bar{u}_1 - \frac{1}{N}u_I + \frac{1}{N}u_J \text{ and }
  \bar{u}_2' = \bar{u}_2 - \frac{1}{N}u_J + \frac{1}{N}u_I.
\end{equation*}

So, 
\begin{equation*}
  \label{eq:7}
  \bar{u}_1'^2 = \bar{u}_1^2 + \frac{2\bar{u}_1}{N}(u_J-u_I) +
  \frac{1}{N^2}(u_J-u_I)^2
\end{equation*}
and
\begin{equation*}
  \label{eq:8}
  \bar{u}_2'^2 = \bar{u}_2^2 + \frac{2\bar{u}_2}{N}(u_I-u_J) +
  \frac{1}{N^2}(u_I-u_J)^2.
\end{equation*}

Since $\sum u_i = 0$, $\bar{u}_1 = -\bar{u}_2$, so
\begin{align*}
  h &= d^2-d'^2 \\
  &= -N \bar{u}_1^2 -N \bar{u}_2^2 + N\bar{u}_1^{'2} + N\bar{u}_2^{'2} \\
  &= 2\bar{u}_1(u_J-u_I) + 2\bar{u}_2(u_I-u_J)+\frac{2}{N}(u_I-u_J)^2 \\
  &= 4\bar{u}_2(u_I-u_J) + \frac{2}{N}(u_I-u_J)^2
\end{align*}

Therefore,
\begin{align}
  \E[h^p] &= \E \left [ \left | 4\bar{u}_2(u_I-u_J) +
      \frac{2}{N}(u_I-u_J)^2 \right |^p \right ] \nonumber \\
  &\leq 2^{p-1} \left ( \E[|4\bar{u}_2(u_I-u_J)|^p] 
    + \E \left [ \left |\frac{2}{N}(u_I-u_J)^2 \right |^p \right ]
  \right ) \nonumber \\
  &\leq 2^{p-1}\left [ (4(b-a))^p \E \left | \bar{u}_2 \right |^p
    + \left ( \frac{2}{N}(b-a)^2 \right )^p \right ] \nonumber \\
  &\leq 2^{p-1} (4(b-a))^p f_{c_1}(p)N^{-p/2} +
  2^{p-1}(2(b-a)^2)^pN^{-p/2}N^{-p/2} \nonumber \\
  &\leq (2^{p-1} (4(b-a))^p f_{c_1}(p)N^{-p/2} +
  2^{p-1}(2(b-a)^2)^p)N^{-p/2} \nonumber \\
  &:= f_{c_3}(p)N^{-p/2} \label{def:hp}.
\end{align}

Now we consider the difference $d-d'$.  For a given $d$ (which grows
linearly), consider $d'$ as a function of the difference:
\begin{equation*}
  d' = \sqrt{d^2-h} = f(h) = f(0) + f'(0) h + \ldots = d -
  \frac{h}{2d} + \ldots
\end{equation*}

The derivative is
\begin{equation*}
  f'(h) = \frac{d}{\sqrt{d^2-h}}
\end{equation*}

By Taylor's theorem, the remainder of the zeroth-order expansion takes
the form 
\begin{equation*}
  R_0(h) = \frac{f'(\xi_L)}{1}h = \frac{-h}{2\sqrt{d^2-\xi_L}}, \quad
  \text{where } \xi_L \in [0, h].
\end{equation*}

Here, we are approximating $d'$ by a constant and bounding the error
by using the first derivative, but it's okay because the square root
function flattens out and the difference inside the square root is
probabilistically small.

Now
\begin{equation*}
  |d-d'| \leq \frac{|h|}{2\sqrt{d^2-\xi_L}} \leq
  \frac{|h|}{2\sqrt{d^2-\max(0, h)}}
\end{equation*}

Recall that $h = d^2 - d'^2$, so 
\[
d^2-\max(0, d^2-d'^2) = 
\begin{cases}
  d^2 & \text{if } d^2-d'^2 \leq 0 \\
  d'^2 & \text{if } d^2-d'^2 > 0
\end{cases}
\]

Therefore, 
\begin{equation*}
  |d-d'| \leq \frac{|h|}{2\min(d, d')} \leq \max \left (
    \frac{|h|}{2d}, \frac{|h|}{2d'} \right ) \leq 
  \frac{|h|}{2d} +  \frac{|h|}{2d'}.
\end{equation*}

The important thing to do is to isolate $|h|$, which is small in
expectation, but not absolutely.
\begin{align}
  \E |d-d'|^p 
  &\leq 2^{p-1} \left ( \E \left | \frac{h}{2d} \right |^p + \E \left |
      \frac{h}{2d'} \right |^p \right ) \nonumber \\
  &\leq 2^{-1} \left ( \E \left | \frac{h}{d} \right |^p + \E \left |
      \frac{h}{d'} \right |^p \right ) \nonumber \\
  &\leq 2^{-1} ( \sqrt{\E[h^{2p}]\E[d^{-2p}]} +
  \sqrt{\E[h^{2p}]\E[d'^{-2p}]} ) \nonumber \\
  &\leq \sqrt{\E[h^{2p}]\E[d^{-2p}]} \nonumber \\
  &\leq \sqrt{f_{c_3}(2p)N^{-2p/2} f_{c_2}(2p) N^{-2p/2}} \nonumber \\
  &\leq \sqrt{f_{c_3}(2p) f_{c_2}(2p)} N^{-p} \nonumber \\
  &:= f_{c_4}(p) N^{-p} \label{def:ddiffp}.
\end{align}

With $q = N\bar{u}_1 - N\bar{u_2} = -2N\bar{u_2}$, and noting that $q$ and $q'$ are
exchangeable,
\begin{align}
  \E[q'^p] 
  &= \E[q^p] \nonumber \\
  &= \E[(-2N\bar{u}_2)^p] \nonumber \\
  &= (-2N)^p \E[\bar{u}_2^p] \nonumber \\
  &\leq 2^pN^p f_{c_1}(p)N^{-p/2} \text{ from } (\ref{def:serfling})
  \nonumber \\
  &= 2^p f_{c_1}(p)N^{p/2} \nonumber \\
  &:= f_{c_5}(p) N^{p/2} \label{def:qp}.
\end{align}

\begin{align}
  \E \left [ \left ( \frac{q'}{dd'} \right )^p \right ]
  &\leq \sqrt{\E |q'|^{2p} \E |dd'|^{-2p}} \nonumber \\
  &\leq \sqrt{\E |q|^{2p} \sqrt{\E |d|^{-4p} \E |d'|^{-4p}}} \nonumber \\
  &\leq \sqrt{\E |q|^{2p} \sqrt{\E |d|^{-4p} \E |d|^{-4p}}} \nonumber \\
  &\leq \sqrt{\E |q|^{2p} \E |d|^{-4p}} \nonumber \\ 
  &\leq \sqrt{f_{c_5}(2p) N^{2p/2} f_{c_2}(4p) N^{-4p/2}} \text{ from
  } (\ref{def:qp}) \text{ and } (\ref{def:dp}) \nonumber \\
  &\leq \sqrt{f_{c_5}(2p) f_{c_2}(4p)} N^{-p/2} \nonumber \\
  &:= f_{c_6}(p) N^{-p/2} \label{def:qpddp}.
\end{align}