Driven by recent advances in the collection of biological data,
many such studies draw from heterogeneous datasources.  We develop
an idea of Jerome Friedman's to conduct two-sample testing using
supervised learning procedures.  In special cases, this technique
generalizes the randomization $t$-test, for which an asymptotic
normality result is known.  Using Stein's method of exchangeable
pairs, we produce Berry--Esseen-type bounds for the permutation
$t$-statistic for the purpose of statistical inference.  We demonstrate
the use of kernel methods in two-sample testing on non-vectorial data
(text and images), and apply multiple kernel learning (MKL) to the
heterogeneous data domain.  We show that these techniques can
effectively synthesize signals from multiple datasources and produce
interpretable weights that highlight the role of each component.